%************************************************
\chapter{Introduction}\label{ch:introduction}
%************************************************

% TODO : fill this
The aim of this project is to develop a platform for a wireless sensor network,
which can be used in various applications, but primarily aimed at home
automation. 

\section{Context}\label{sec:context}

First, talk about social implications of home automation, how the market is
evolving, what kind of tech is around.

% TODO : write this section
Second, look at what kinds of tech are available, mention how fast the industry
is moving, and therefore why it's worth surveying the landscape regularly to
scout out new solutions and test them. Take some established products and show
how they have been out-specced by more recent chips, only a few years later. 

Atmel chips are a good example, with recent designs having better energy
consumption, higher output power, multi-antenna designs, etc.

\section{Cahier des Charges}\label{sec:cdc}

We will begin with a brief description of the project's requirements. These are
the guidelines that were set forth at the beginning of the project. This list of
items should cover the needs of a basic sensor network for indoor use. This
section only provides an overview of the various subjects that will need to be
covered. Each one will be looked at more in detail later in this document.

\begin{itemize}
  \item \emph{Low production cost} :
    the system must be scalable within
    a domestic environment and therefore remain very affordable. The aim is for
    the main components to cost less than CHF~10. The cost of the final version
    will be analyzed in \autoref{sec:cost}.
  \item \emph{Ultra low consumption} :
    wireless modules must be able to run for a very long time on a single
    charge. We shall also attempt to verify whether it is possible for the
    device to run using only indoor solar power.  Power consumption will be
    covered in \autoref{sec:power}.
  \item \emph{Bidirectional RF} : 
    in order to work as a network, all devices must be able to communicate both
    way. This will allow at the very least for an end device to acknowledge
    that it has correctly received an instruction.
    Choices in RF technology will be explained in \autoref{sec:frequency}.
  \item \emph{Processing power} :
    the basic platform must be able to run some algorithms on live audio data,
    and therefore must be capable of calculating a basic \ac{FFT} in real time.
    This requirement will be taken into account when comparing various
    microcontrollers, in \autoref{sec:chips}.
  \item \emph{I/O} :
    each basic module should have some spare inputs and outputs which can be
    used to drive LEDs or relays (using \ac{PWM} if necessary) and gather input
    from switches or sensors.
  \item \emph{Analog Inputs} :
    the device must have at least one analog input. Digital and analog outputs
    will be described in \autoref{sub:io-connector}.
  \item \emph{Storage} :
    the device must have sufficient memory to run the algorithms necessary for
    the target application.\footnote{As the target application had not been
    clearly defined when the project requirements were drafted, this is quite
    a subjective requirement.}
\end{itemize}

The goal of this project is to evaluate the various wireless communication
standards that are currently on the market, and produce a working proof of
concept for a very basic communication between two nodes. It will concentrate
mostly on researching existing solutions and producing the necessary hardware to
evaluate the platform. The software will not play an important part in this
stage of the project.

\section{Power}\label{sec:power}

One of the selling points of a home automation network is that it should not get
in the way. After it has been installed, it should require as little maintenance
as possible. If a user were to install several dozen wireless modules throughout
their home, and each of these modules required a battery change every year, the
user could end up having to replace batteries on average several times a month. 

It is not always possible to connect a module to a mains power source. For
instance, some modules will be placed outside the house to measure temperature
and humidity. Some could be placed on or under tables, to monitor activity, or
in various places along walls where wiring them would be impractical. For this
reason, most of the modules forming this network should be able to function
using only solar power harvested inside the house, with a battery to store
excess power.

In order to meet this goal, we will develop a module which will be capable of
spending most of its time in sleep mode, using only a low-powered clock to wake
up periodically. We are hoping for power consumption below
\SI{1}{\micro\ampere} in sleep mode. In parallel to this project, a series of
tests will be run, logging the amount of power that can be harvested from indoor
illumination using small and cheap solar panels.

In spite of these observations, these modules were primarily targeted towards
audio processing applications. The project specifications demanded that at least
one amplifier be present on one of the analog inputs. These components demand
a certain current to function. As an example, the MAX4466 micropower
preamplifier, included in the early revisions of the design, consumes
\SI{24}{\micro\ampere} constantly when enabled. This was taken to account when
choosing the microcontroller for our application. More details follow in
\autoref{sec:chips}.

Finally, not all nodes will necessarily be solar powered. In order to form
a reliable network, some devices must act as ``routers''. In mesh networks such
as ZigBee or \ac{6LoWPAN}, these devices will often be constantly powered on,
ready to receive and relay information to the rest of the network.

% TODO : either go into more detail here, or later in a network section.
% otherwise link to relevant documentation
