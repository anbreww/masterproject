%************************************************
\chapter{Design Choices}\label{ch:choices}
%************************************************

\section{Communication}\label{sec:communication}

The method of communication is the main specification which must be decided
before we can choose a processing platform and start working on the design of
the PCB. Since the design is targeted at home automation, we have decided that
the following features would be desirable to create a useful platform :

\begin{itemize}
  \item \emph{Universal} :
    The frequency used to communicate must be universal and available worldwide,
    so that the same hardware can be used anywhere without the need to develop
    region-specific versions. Emission power must also comply to regional
    regulations in all areas where the design is to be used.

  \item \emph{Popular network stack} : 
    The network stack should ideally be a popular industry standard. This will
    likely ensure that it is supported by more vendors, leading in turn to more
    available libraries, more documentation and more support. By using
    a well-established standard, we can ensure that if one vendor drops their
    support, there will still be many others to choose from.

  \item \emph{Interoperability} : small sensor devices are not very useful on
    their own. It is important to be able to extend the network. The best way to
    achieve this, in the author's opinion, is to rely on \emph{open standards},
    which would allow devices from multiple vendors to communicate using
    a common set of standards and protocols. 

  \item \emph{Reachable from anywhere} : if the application requires it, each
    node must be reachable from anywhere the user may be. It is important for
    the sensor network to be able to interface with the global internet if need
    be. More details about this can be found in \autoref{sub:ipv6}.

  \item \emph{Mesh topology (or similar)} : This can drastically improve the
    range of a sensor network while keeping infrastructure costs low. See
    \autoref{sec:topology} for more details.
\end{itemize}


%\begin{quotation}
%Although the differences in radio specification have some effect, it is
%nothing like as great as that of the higher-layer stack.
%
%    --- \defcitealias{hunn2010}{Essentials of Short-Range Wireless}
%    \citetalias{hunn2010} \citep[pg. 20]{hunn2010}
%\end{quotation}


\subsection{IPv6}\label{sub:ipv6}

IPv6, which stands for Internet Protocol Version 6 si the next standard that
will be used for communication between networked devices around the world. It
was designed to replace the current IPv4, which most people will be familiar
with at the moment.\footnote{For more information, see \url{
http://en.wikipedia.org/wiki/IPv4_address_exhaustion }.}

One of the key differences between the two versions of the protocol is the size
of the address pool. IPv4, which most people will be familiar with at this time,
defines the address of a device as 32 bits, usually represented as four groups
of eight digits, such as in the form \texttt{128.178.50.12}. Unfortunately, with
the growing number of internet-connected devices, the pool of available
addresses is steadily running out. IPv6 is an answer to this concern (among many
others), as it provides $2^{128}$ possible addresses, namely $3.4 \cdot 10^{38}$
addresses. 

The technical details of these protocols are beyond the scope of this project.
We will refer the reader to the innumerable information sources available on the
web, as well as the book ``Migrating to IPv6'' by Marc
Blanchet\citep{blanchet2006}, which explains these issues in detail.

A discussion on the use of IPv6 for sensor networks can be found in the book
``Interconnecting smart objects with IP''\citep[chap. 4]{dunkels2010}, under the chapter
\emph{IPv6 for Smart Object Networks and the Internet of Things}.

What we shall retain, in the context of this project, is that by using this new
protocol, we can provide every device with its own indivual address, allowing it
to be reachable over the global internet. This will allow us to poll sensors
individually or send commands to devices from anywhere in the world, with no
need for a local application-specific gateway to translate commands between the
embedded devices and the outside.

This is one of the promises of 6LoWPAN, which we will cover later in the
discussion of protocol stacks, in \autoref{sub:6lowpan}



\section{Topology}\label{sec:topology}

Wireless devices have the possibility to form links with any other device within
their range. It is therefore important to define how these devices will
communicate together : whether we will simply link two devices together in
a point-to-point fashion (essentially replacing a wire with a radio link), to
broadcast from one device to all receivers within range, or to use a more
complex topology.

There are many possibilities available\footnote{for more examples of topologies,
see \citep[sec. 2.3.4]{hunn2010}.}, but we will only cover the most popular
option : mesh topology.

One of the differentiating factors of the IEEE 802.15.4 standard is the ability
to form a mesh network. This is one of the most complex topologies to implement,
but luckily its operation is covered by existing user libraries already, which
makes it an excellent choice for developing robust applications.

In a mesh network, nodes are differentiated by their role. In this case, there
are two possibilities. \emph{End nodes} are low-powered nodes that spend most of
their time asleep. In some cases, they will wake up periodically to check for
any pending messages. In others, they will remain asleep until they have data to
transmit. These devices are interconnected by \emph{routers} : nodes that are
always awake, ready to relay messages between other nodes.

\begin{figure}[htb]
  \begin{center}
    \includegraphics[width=0.7\textwidth]{gfx/snippets/mesh}
  \end{center}
  \caption{Mesh topology\citep{hunn2010}}
  \label{fig:mesh-topology}
\end{figure}

\autoref{fig:mesh-topology} illustrates the way that nodes are connected
together. The white circles are end nodes, whilst the grey circles represent
routers. As their name suggests, \emph{end} nodes are only meant to be at the
edges of the network and will not relay messages. The black circle is a special
kind of router node, called the \emph{coordinator}. This node handles the
creation and management of the network and must always be present.
\marginpar{note: IEEE~802.15.4 also allows the creation of a star topology, if
a mesh network is not needed.}

In practice, this topology has two important advantages.

First, it allows the network to be extended with very little infrastructure
cost. Indeed, if two end nodes are too far apart, all that is needed is to place
one or more router nodes in between them. These will then automatically relay
data between the neighbour nodes. For example, it is likely that in a domestic
environment two nodes will only achieve a range of one storey vertically. If the
network required fixed infrastructure (as is the case for a WiFi network for
instance), a dedicated access point would have to be installed at least every
other storey. However, in the case of a mesh network, the router nodes on each
floor will be able to relay messages between the floor above and the floor
below.

Second, having multiple devices in the same area will increase the robustness of
the network. When a message is to be transmitted, nodes will auto-discover the
most efficient path to deliver it. If some nodes along that path are moved
around or fail, the network will re-route to avoid the faulty paths. 


\section{Operating frequency}\label{sec:frequency}

The first choice to be made is the frequency of operation. This is quite an
important topic as it will impose restrictions on the type of networking
protocols that will be available, so long as we wish to comply with current
standards. Some frequency bands can cause interference with common household
equiment more than others. Furthermore, some choices also limit the target
markets : some frequency bands are not available in all countries around the
world or require special licenses.

We will cover only the most common choices and explain our reasoning for each
one. Since the devices under development are to be used in a domestic
environment by a regular consumer, they must be usable without a specific radio
license. It would also be preferable if the devices did not have to be modified
according to the market they will be deployed in. The frequency choices are
grouped in the \ac{ISM} band, which makes them possible to use license-free, but
also subject to interference from other household devices such as microwave
ovens.

\marginpar{Note: the \SI{5}{GHz} band will be omitted as it is currently
used only for WiFi, from the IEEE~802.11n standard and further}
At the moment, the most popular choices for wireless networks are the so-called
``Sub-GHz'' band, and the 2.4 to \SI{2.5}{GHz} band, which will be described
below.

\subsection{Sub-GHz}

The main appeal of the sub-GHz band is that it is less crowded than the
\SI{2.4}{GHz} band. This is becoming increasingly true with time as we are
seeing more and more devices in this part of the spectrum. Laptop computers,
home audio systems, smartphones all contribute to noise in this band, which is
shared between Wi-Fi, Bluetooth, ZigBee and many other technologies. Therefore
it can be quite attractive to develop a sensor network which functions outside
of this frequency range, in hopes that it will be more reliable.

Currently, the main disadvantage of this frequency range is the lack of
worldwide consensus on frequency allocations. The \ac{ITU} divides the world
into three regions, for managing the radio spectrum:

\begin{itemize}
  \item \emph{Region 1} : Europe, Africa, the Middle East and most of the
    ex-Soviet Union.
  \item \emph{Region 2} : North and South America, Greenland and some of the
    Pacific Islands.
  \item \emph{Region 3} : most of Asia and Oceania.
\end{itemize}

These three regions are illustrated on \autoref{fig:itu-regions}, taken from the
Swiss website of the \ac{ITU}

\begin{figure}[htb]
  \begin{center}
    \makebox[\textwidth][c]{
      \includegraphics[width=1.2\textwidth]{gfx/itu-regions}
    }
  \end{center}
  \caption{ITU Regions [itu.ch]}
  \label{fig:itu-regions}
\end{figure}

\subsubsection{\SI{433}{MHz}}

The \SI{433}{MHz} ISM band (\SIrange{433.05}{434.79}{MHz})
 is a very popular choice for short range,
low cost devices. Its use is mainly for remote controlled devices, such as light
switches or garage door openers. It is usually used for one-way transmissions.

In Switzerland, the power limit for \SI{433}{MHz} is \SI{1}{mW} at 100\% duty
cycle, or \SI{10}{mW} at 10\% duty cycle.\footnote{National Frequency Allocation Plan
\url{http://www.bakom.admin.ch/themen/frequenzen/00652/00653/index.html?lang=en}}

According to the ITU, this frequency band is only available to Europe and the
Middle East (Region 1). However, the FCC grants a regional exception that allows
the \SI{433}{MHz} band to be used in the United States (Region 2).

Currently though, there doesn't seem to be any widespread use of networking
stacks operating at this frequency.\footnote{There is no technical obligation to
use a stack explicitly designed for a given frequency, but doing so would limit
the ability to interact with other devices following the same standard.} The
IEEE has only recently amended the 802.15.4 standard to add support for this
frequency range.\footnote{IEEE~802.15.4c\texttrademark{}-2009 : Amendment 2 :
Alternative Physical Layer Extension to support one or more of the Chinese
314-316~MHz, 430-434~MHz, and 779-787~MHz bands.}

In \autoref{tab:frequency-comparison}, the ``Y/N'' notes in region independence
and availability of networking stacks refer respectively to the facts that,
\begin{itemize}
  \item Region independence is not enforced by the ITU but rather enabled by
    regional exceptions.
  \item Mesh networking stacks are available, but recent and therefore not yet
    widely used.
\end{itemize}

\subsubsection{868 / \SI{915}{MHz}}

The second option that we will consider in the sub-GHz band is around
\SI{900}{MHz}. Unfortunately, it operates in slightly different frequency ranges
depending on the target region. In \ac{ITU} Region 1, it is centered around
\SI{863.5}{MHz}, whereas in ITU Region 2, the devices will operate around
\SI{915}{MHz}.

This choice is becoming more common and is being heavily marketed by silicon
chip makers as an alternative to \SI{2.4}{GHz} for mesh networking. The
IEEE~802.15.4 in its 2003 version allows for 10 communication channels in the
US, and only one channel in Europe. The 2006 revision extends the number of
channels to 30 for the US (The European version remains unchanged).

Devices operating in this frequency range are becoming increasingly common and
it seems now that most major chip manufacturers are producing chips that can
switch between these bands ( 433 / 868 /\SI{915}{MHz}).\footnote{One example is
Texas Instruments' CC1101 : \url{http://www.ti.com/product/cc1101}.}

\subsection{\SI{2.4}{GHz}}

This band is currently the most popular part of the spectrum for personal
electronic devices that must communicate together. Current applications include
WiFi, Bluetooth, and the IEEE 802.15.4 standard, which is the basis for ZigBee
and \ac{6LoWPAN}, which will be covered in \autoref{sub:6lowpan}.

This popularity has its advantages as well as drawbacks. One of the main
advantages is that there are a very wide variety of devices now available that
operate at this frequency. 

In spite of this caveat, the author's previous experience with ZigBee
devices\footnote{tests were done using Digi XBee Series 2 transceivers, using
1/4 wave whip antennas, but will not be documented in this report.}
has shown that in a typical domestic environment, with many parasitic wireless
devices active, it is possible to achieve a reliable link (less than 5\% packet
loss) between devices in different rooms and on different floors : two devices
were able to communicate between the basement and a room two floors above.
Furthermore, we will be favouring a networking stack that can provide mesh
capabilities, as explained in \autoref{sec:topology}. This will lower the
importance of the range, as extending the range of the network is a simple
matter of adding a cheap repeater (router) device at some point in the data
path. For example, if communication cannot be established between the first and
third floors, a router can be placed on the second floor to bridge the gap
between those two domains.


\section{Comparison}

To sum up this section, all three frequency options are compared in
\autoref{tab:frequency-comparison}. One criteria was added which was not
discussed up to this point : antenna size.

Although regulations on emission power and bands change from region to region,
the size of the antenna is directly linked to the operating frequency. This is
important to note, as a larger antenna will require more board space, or
a longer piece of wire, which will take more space inside or around an
enclosure.

For this comparison, we used a quarter wave antenna, whose length can be
calculated with the following formula :

\begin{equation*}
  L = \frac{1}{4} \cdot \frac{c}{\nu},
\end{equation*}

where

\begin{tabular}[h!]{ll}
$c$   & speed of light  \\
$\nu$ & carrier frequency \\
\end{tabular}

\begin{table}
    \myfloatalign
  \begin{tabularx}{\textwidth}{Xlll} \toprule
    \tableheadline{Criteria}
    & \tableheadline{433 MHz}
    & \tableheadline{868 / 915 MHz}
    & \tableheadline{2.4 GHz} \\ \midrule
    Region Independent    & Y/N           & N             & Y \\
    Mesh networking stack & Y/N           & Y             & Y \\
    1/4 wave antenna      & \SI{17.3}{cm} & \SI{8.3}{cm}  & \SI{3.06}{cm} \\
    \bottomrule
  \end{tabularx}
  \caption[Frequency comparison]{Frequency comparison}
  \label{tab:frequency-comparison}
\end{table}


\bigskip
Based on the facts described in the previous sections, we chose to favour
a \SI{2.4}{GHz} solution for our design. This guarantees that the same product
can be used in any region in the world, as well as requiring the smallest
antenna. At design time, there were also more 802.15.4-compatible chips
available on the market for \SI{2.4}{GHz} than for any other operating
frequency, as can be seen from the comparisons in \autoref{sec:chips}.

\section{Network stacks}\label{sec:stacks}

Before choosing a hardware solution, we must decide how we wish to communicate
at the software level. If our communication standard acts as a simple
remplacement for a serial connection, sending commands is a relatively trivial
matter, requiring only to send and receive minimal amounts of data (a few
characters) over a serial interface.  If we wish to do anything more complex,
such as mesh networking, we will have to use a complete networking stack, or
develop our own.

Since developing our own network stack is beyond the scope of this project, we
will rely instead on some well-established standards. There is a large amount of
standards to choose from, so we will only present a few candidates and explain
the reasoning behind our choices.

It is interesting to note that although there are many software solutions
available, most are based on one of two standards for the lower-level
specifications (the physical layer, known as \emph{PHY}, and the link layer,
referred to as \emph{MAC}, the bottom two layers of the OSI model\footnote{for
more information : \url{http://en.wikipedia.org/wiki/OSI_model}.}).

The IEEE~802.15 working group focuses on \ac{WPAN}s. Bluetooth uses the
IEEE~802.15.1 standard, while almost all other stacks referred to here are based
on IEEE~802.15.4. This implies that unless constrained by licensing requirements
or physical memory, most of the common networking stacks should be
interchangeable. As we will see, this does not necessarily mean that all stacks
are portable across devices, as some are proprietary and only available on
certain architectures.

\subsection{Bluetooth Low Energy}\label{sub:bluetooth}

Bluetooth 4.0 is the latest iteration of the Bluetooth specification. Although
its previous incarnations were not very well suited to being used in sensor
networks (partly because of their power requirements, and relatively long
start-up times.)\footnote{Bluetooth 2.1 set-up time can be up to 6 seconds,
while Bluetooth 4.0 promises times under \SI{3}{ms}. }

Although Bluetooth 4.0 seems like a promising technology, and is already being
integrated into consumer products such as the iPhone 4S, at the time we looked
into wireless networking standards for this project, chips compatible with this
standard were not yet widely available. This option was therefore put to the
side, but would be worth considering for a new project.

\subsection{ZigBee}\label{sub:zigbee}

ZigBee is a very well-established industry standard for wireless sensor
networks.\footnote{the ZigBee alliance (\url{http://www.zigbee.org}) has over
300 industy members.}

It is based on the IEEE~802.15.4 physical standard, and adds several
useful features such as application profiles and security. It supports several
topologies, including mesh networking.

The main advantage of this standard is that it is very popular, and therefore
many chip vendors propose a ZigBee stack compatible with their products. This,
combined with application profiles, means that in theory it is possible to
connect multiple devices together from different vendors and ensure that they
communicate with the same standard.

The biggest disadvantage of the ZigBee protocol stack is that it requires
a license for commercial use. Some vendors, such as Microchip\footnote{\url{
http://www.microchipdirect.com/ProductSearch.aspx?Keywords=SW183053 }} require
a payment in order to use the stack which is compatible with their
devices.\footnote{at the time of writing, Microchip's ZigBee stack sells for
\$995.}

\subsection{6LoWPAN}\label{sub:6lowpan}

\ac{6LoWPAN} is an open standard, developed by the \ac{IETF}, which aims to create
a bridge between low-powered devices and the global internet. It is built on the
IEEE~802.14.5 standard, and supports multiple topologies, including mesh
networking. 

One of the distinguishing features of 6LoWPAN is that instead of using
a proprietary format for data packets, it uses a compressed version of IPv6
headers. This enables any 6LoWPAN network to be connected to the global internet
by using a so-called \emph{edge router} : a device which will bridge both
networks together by compressing/decompressing packets and forwarding them to
the relevant interface.\footnote{for more details, we recommend \citep[sec.
6.4]{shelby2010}. For a more thorough introduction to 6LoWPAN, \citep[chap.
16]{dunkels2010} is a good start.} The advantage of this solution is that it
   requires little overhead and is therefore well suited to cheap and
   low-powered devices with limited processing power. Only one edge router is
   required to connect one cluster of devices to the Internet.

Although vendor support is not yet as good as for ZigBee, many open source
projects have support for 6LoWPAN on various architectures. The Contiki
OS\footnote{ \url{http://www.contiki-os.org} } project is an example that is
often used in academic research. 

Unllike ZigBee, the 6LoWPAN standard itself does not specify application
profiles. However, several projects are under way to develop an open standard
for application profiles. The standard currently under development by the IETF
is named \ac{CoAP}\footnote{ \url{
http://tools.ietf.org/html/draft-ietf-core-coap-08 }}


\subsection{Z-Wave}\label{sub:z-wave}

Z-Wave is a popular\footnote{the Z-Wave Alliance (\url{http://www.z-wavealliance.org})
currently comprises over 160 members.} choice in home automation, so we feel it
is worth mentioning here. However, it is a completely proprietary standard,
which can only be used after signing a \ac{NDA} with the original developer,
Zensys.\footnote{ \url{ http://www.zen-sys.com } }

Z-Wave uses sub-GHz bands (specifically around \SI{900}{MHz}), and uses
a proprietary hardware link layer.


\subsection{MiWi}\label{sub:miwi}


MiWi is a lightweight proprietary protocol stack developed by Microchip for
their range of microcontrollers. It is based on the same hardware layer as
ZigBee and 6LoWPAN (IEEE~802.15.4), but is not compatible with either of these
standards. 

Although it can be downloaded for free from the Microchip website\footnote{ \url
{http://www.microchip.com/MiWi} }, its license only allows it to be used on
Microchip microcontrollers.

\subsection{ANT}\label{sub:ant}

ANT is a proprietary solution that operates at \SI{2.4}{GHz}, designed for
wireless sensors and low-power operation which is becoming popular in sports and
monitoring devices.  Unfortunately, it is completely proprietary (much like
Z-Wave), and at present only two manufacturers offer ANT devices\footnote{ \url{
http://www.thisisant.com/pages/products/chip-based-solutions } } : Nordic
Semiconductor and Texas Instruments. Due to its proprietary nature
and low adoption from chip manufacturers, it was not considered in this project.

\subsection{Network Stacks : summary}

\autoref{tab:stack-comparison} provides a quick summary of some of the features
of the stacks discussed in this section.  For the purpose of this project, we
wish to design a platform which will be future-proof as much as possible. To
achieve this, we will try to avoid relying on proprietary standards and
licensing complications. 

\begin{table}[h]
    \myfloatalign
    \makebox[\textwidth][c]{
  \begin{tabularx}{1.4\textwidth}{Xllllll} \toprule
    \tableheadline{Criteria}
    & \tableheadline{ZigBee}
    & \tableheadline{BT LE}
    & \tableheadline{6LoWPAN}
    & \tableheadline{MiWi}
    & \tableheadline{Z-Wave}  
    & \tableheadline{ANT} \\ \midrule
  Mesh           & Y           & Y        & Y        & N           & Y           & Y \\
  Star           & Y           & Y        & Y        & Y           & -           & Y \\ 
  License        & Proprietary & -        & Open     & Proprietary & Proprietary & Proprietary \\ 
  Cross-Platform & Y           & -        & Y        & N           & Somewhat    & N \\ 
  Radio          & 802.15.4    & 802.15.1 & 802.15.4 & 802.15.4    & Prop. 900MHz& Proprietary \\
    \bottomrule
  \end{tabularx}
  }
  \caption[Network stack summary]{Network stack summary}
  \label{tab:stack-comparison}
\end{table}


We have chosen to narrow down our choice to ZigBee or 6LoWPAN and use a hardware
device that can support both. Since both protocols share the same hardware
requirements, as well as similar memory footprints, it is quite possible to
develop a platform where ZigBee or 6LoWPAN are interchangeable. This would allow
us to experiment either with integrating with existing devices, or develop
IPv6-compatible sensors for the future ``Internet of Things''.

Our first choice will therefore go towards a hardware solution that provides an
IEEE~802.15.4-compatible MAC and PHY over \SI{2.4}{GHz}.

\section{RF Chips}\label{sec:chips}

Once the choice was made of which operating frequency and protocol stack would
be used, the manufacturer's data on compatible chips was compiled, to allow us
to compare all possibilities easily. 

These comparison tables can be found in \autoref{ch:appendix-comparisons}. The
first table contains a list of transceivers, while the next table, split into
two parts, contains a list of system-on-chips.

\subsection{Multi-chip solution vs. System-on-Chip}

In our comparison, we will note two different categories of chips. The first
category, which we will refer to simply as ``transceivers'', provide a wireless
transceiver on an external interface. Mostly, these will manage the MAC and PHY
of the hardware standard. The rest of the stack (from network layer to
application layer)\footnote{For readers unfamiliar with the OSI model, more
information can be found here : \url{ http://en.wikipedia.org/wiki/OSI_model }.}
must be managed by an external microcontroller. 

This means that a design using a transceiver chip requires at least two hardware
chips : the main application microcontroller and the transceiver.

The second category contains system-on-chips. These devices integrate
a transceiver and a microcontroller in the same package. Each solution has
advantages and disadvantages.

\paragraph{Board space} 

Since the system-on-chip (SoC) only needs one physical package, it usually
occupies less board space. If the application calls for a very tight design, it
would be preferable to opt for a SoC to save space.

\paragraph{Cost} 

SoCs cost more than transceivers, but transceivers require an additional
microcontroller to function. As a result, we can expect the cost of the
microcontroller+transceiver combo to be fixed if we choose a SoC, whereas in the
case of separate components, the cost will vary depending on the microcontroller
that is chosen for the application

\paragraph{Flexibility} 

The biggest drawback with an SoC is the lack of flexibility that it imposes on
the design. If we wish to use a SoC, we must ensure that it posesses enough
processing power for all potential applications, as changing the architecture
may require a complete redesign.  If the system is made of two seperate parts,
it is possible to replace the microcontroller without modifying the RF part of
the design.

\paragraph{Complexity} 

From a developer's point of view, some solutions are easier to develop for than
others. On a system-on-chip for instance, all software layers are present on the
same device. This can make it difficult to separate the code for low-level
networking features from the application level code.

Some chips, such as the CC2520, handle the 802.15.4 MAC layer so that the
application chip can focus on higher level code. Other solutions, such as the
XBee integrated devices from Digi\footnote{ \url{ http://www.digi.com/xbee/ }}
handle the complete ZigBee stack, and only require the application to create
frames with data and a target address.

To address this issue in a system-on-chip solution, one possibility is to
program the devices with a network stack and bootloader, and transmit only
application-level code to the devices. This allows the final software developer
to develop network applications without having to compile any code related to
the low-level networking functions. In this regard, the Jennet-IP protocol stack
seemed promising. Unfortunately, it was to be released in the fourth quarter of
2011\footnote{ \url{
http://www.eetimes.com/electronics-news/4216038/Home-or-commercial-lighting-comes-under-Internet-control}
} under an open source license, but it is still not available to this day.

\subsection{Final choice}

In the end, the solution selected for this application was the STM32W108
System-on-Chip, for the following reasons.

\paragraph{Architecture}

the STM32W108 is based on a 32-bit ARM core, which
operates at a frequency up to 24 MHz. It is the only solution in this comparison
to offer a single-cycle Multiply and Accumulate instruction (see
\autoref{tab:soc-comparison}).

\paragraph{Peripherals} 

the STM32W series includes an analog comparator which
can trigger device wake-up (which was required as part of the project
specifications), as well as a 6-channel \ac{DMA}, which helps speed up
calculations and lessen the load on the processing core by automating some read
and write operations.

\paragraph{RF Performance} 

RF performance and associated power consumption are
on par with Atmel's ATMEGA128RFA1 and Texas Instruments' CC2530. Specs given in
the comparison table do not take into account ``boost mode'', which is said to
add approximately \SI{1}{dBm} to the receiver's sensitivity and \SI{0.5}{dBm} to
the transmitter.

The choice between the ATMEGA128RFA1, CC2530 and the STM32W was made mainly on
the grounds of architectural differences and price.

The ATMEGA, being an 8 bit microcontroller with no MAC instruction or DMA, would
not be sufficient to fulfill the requirement for an FFT calculation. If this was
not required by the target application, this would be an interesting choice.
The CC2530 does include a 5-channel DMA , but no MAC instruction. It is however
more expensive than the STM32W108. 

\section{Microcontroller}

The chosen microcontroller, the STM32W108 system on chip, is based on a ARM
Cortex M3 core, a 32-bit architecture that is licensed out to many different
chip manufacturers. The chip uses the Thumb-2 instruction set, and can be
programmed in C using the gcc-arm compiler. Open source code and compilers are
readily available for this platform. However, the radio frequency section of the
design is not documented, and subject to a non-disclosure agreement. All
interactions with the radio layer of the device must go through the pre-compiled
software library provided by the manufacturer.

The STM32W108 is a compromise : we put aside some more economical or more
power-efficient architectures in favour of the additional processing power that
was imposed by the project specifications.
More complex applications may call for more processing power than this chip can
provide. If this is the case, then a more powerful microcontroller would have to
be added to the design, and the STM32W108 would be reduced to serving simply as
a network interface, while the core application would run on the main processor.
In this case, the additional cost justified by the Cortex M3 core would be
wasted, as a less powerful microcontroller would be sufficient to handle the
networking tasks. This is a point to consider in any future revisions of this
project.
