%************************************************
\chapter{Design Choices}\label{ch:choices}
%************************************************

\section{Operating frequency}\label{sec:frequency}

The first choice to be made is the frequency of operation. This is quite an
important topic as it will impose restrictions on the type of networking
protocols that will be available. Some frequency bands can cause interference
with common household equiment more than others. Furthermore, some choices also
limit the target markets : some frequency bands are not available in all
countries around the world or require special licenses.

We will cover only the most common choices and explain our reasoning for each
one. Since the devices under development are to be used in a domestic
environment by a regular consumer, they must be usable without a specific radio
license. It would also be preferable if the devices did not have to be modified
according to the market they will be deployed in. The frequency choices are
grouped in the \ac{ISM} band, which makes them possible to use license-free, but
also subject to interference from other household devices such as microwave
ovens.

\subsection{433 MHz}

The 433 Mhz ISM band (433.05 Mhz - 434.79 Mhz)
\footnote{\url{http://www.bakom.admin.ch/themen/frequenzen/00652/00653/index.html?lang=en}
National Frequency Allocation Plan} is a very popular choice for short range,
low cost devices. Its use is mainly for remote controlled devices, such as light
switches or garage door openers. It is usually used for one-way transmissions.

In Switzerland, the power limit for 433 Mhz is 1mW at 100\% duty cycle, or 10mW
at 10\% duty cycle.

% TODO : find the OFCOM document with power limits per frequency band

Although quite common in Europe for cheap home automation
devices, it is not usable in the United States (known as \emph{Region
2})\footnote{For more details, refer to the documentation about the
  International Telecommunication Union regions
  : \url{http://life.itu.ch/radioclub/rr/art05.htm}.}.
Therefore, there are few commercial solutions and chips available using this
frequency. At the time of writing, no widespread networking solution was found
using this particular frequency band.  Furthermore, any developments made using
such a device would have to be repeated in another frequency band if the devices
were to be used outside of Europe and the Middle East (\emph{Region 1}).

% TODO : why not make our own solution? explain.

\subsection{Sub-GHz}

In addition to the 433 Mhz ISM band, there is another option, commonly referred
to as the \emph{Sub-Ghz} band, which is centered around 863.5 Mhz in Region 1,
and 915 Mhz in Region 2.

% TODO : reword the following paragraph
Devices operating in this frequency range are becoming increasingly common. Most
major chip manufacturers are producing chips that can switch between these bands
( 433 / 868 / 915 Mhz ).

The main appeal of the sub-Ghz band is that it is less crowded than the 2.4 Ghz
band. This is becoming increasingly true with time as we are seeing more and
more devices in this part of the spectrum. Laptop computers, home audio systems,
smartphones all contribute to noise in this band, which is shared between Wi-Fi,
Bluetooth, ZigBee and many other technologies. Therefore it can be quite
attractive to develop a sensor network which functions outside of this frequency
range, in hopes that it will be more reliable.

The greatest disadvantage of this frequency range is the lack of a common band
between regions. If a device is developed for use in Europe, a different version
will have to be made for the North American market.

\subsection{2.4 GHz}

This band is currently the most popular part of the spectrum for personal
electronic devices that must communicate together. Current applications include
WiFi, Bluetooth, and the IEEE 802.15.4 standard, which is the basis for ZigBee
and 6LoWPAN, which will be covered elsewhere in this report.

This popularity has its advantages as well as drawbacks. One of the main
advantages is that there are a very wide variety of devices now available that
operate at this frequency. 

% TODO : define ``elsewhere''

\section{Comparison}

Here, a table comparing the advantages/disadvantages of each tech

\section{In hindsight}

What have we learned? Would it have been better to choose a different frequency?

\section{Network stacks}\label{sec:stacks}

A table comparing ZigBee (various versions), Z-Wave, Bluetooth, 6LoWPAN etc
would be nice.
Explain the mostly subjective reasons for choosing a solution based on open
standards, IETF approved drafts etc. 

Explain that the choice between ZigBee and 6LoWPAN is more subjective than
technical, and that it's difficult to predict where the market will go in the
near future, and that the technology for the internet of things has not been
decided yet. However, it is very likely that it will work only if it is based on
open standards and if everyone collaborates. This is why we chose IEEE 802.15.4
on 2.4 GHz, because the same hardware would allow us to make a ZigBee or 6LoWPAN
compatible network.

\section{RF Chips}\label{sec:chips}

Here, we'll talk about the various microcontrollers on the market

\subsection{Multi-chip solution vs. System-on-Chip}

Do a cost comparison. 
Explain technical advantages (notably space saved on PCB) and disadvantages
(software is more complex, less separation of application vs stack).

Mention some advantages like the over-the-air bootloader for STM32 and the
Jennet-IP solution (and why we couldn't use it!) \footnote{Jennet IP was to be
released under an open source license in the fourth quarter of 2011
: \url{http://www.eetimes.com/electronics-news/4216038/Home-or-commercial-lighting-comes-under-Internet-control}
but it's still not available today}

Now we'll want to somehow format the huge excel table into something we can
include in a report\ldots hmmm.

\section{Observations}

What we've deduced from the project : was it the right choice? What could we
have done better?

\begin{itemize}
  \item \emph{Power consumption} : original specifications called for an audio
    amplification stage, which would have used tens of micro Amps.
    % TODO : get proper data
    Therefore a difference between 400nA and 20nA was not significant. However,
    if the system is designed to be completely turned off most of the time, then
    a combination of an ultra low power microcontroller with a CC2420 or atmel
    chip would be much more efficient.
  \item \emph{Antenna diversity} : performance in the 2.4 Ghz band is pretty
    random (show some examples!). Very sensitive to positioning and orientation.
    Some chips now offer antenna diversity (link to very recent Atmel chip),
    which could be a very interesting option if space permits. Especially if
    using PCB antenna, would not increase costs too much.
\end{itemize}
