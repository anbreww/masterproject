%*****************************************
\chapter{Hardware development}\label{ch:examples}
%*****************************************
%\setcounter{figure}{10}

The hardware for this device went through two major revisions. The first was
destined to assess the functionality of the chip, and make sure that it was
possible to establish communication between two nodes. Since design of a system
with a built-in antenna and transceiver was new to the author, it was essential
to proceed in small, manageable steps to help debugging if problems were to
arise. For these same reasons, the current designs remain very close to the
suggested reference designs. \autoref{fig:base-platforms} shows both versions of
the device side-by-side.

After testing several versions and multiple components, and comparing them to
existing designs with other chips, we have formed some ideas on ways to improve
the design, which we will cover in \autoref{ch:improvements}.
Indeed, although we have obtained performance very similar to the reference
design provided by ST Microelectronics, we believe it is possible to achieve
a better design than the reference. See \autoref{ch:performance} for the
results.

% TODO : images here
\begin{figure}[bth]
  \myfloatalign
  \subfloat[Version 1]
  {\includegraphics[width=.45\linewidth]{gfx/base-node_photo_lores}}
  \subfloat[Version 2]
  {\includegraphics[width=.45\linewidth]{gfx/frontpage}}
  \caption{Base platforms}
  \label{fig:base-platforms}
\end{figure}


\section{Version 1}

Version 1 is comprised of the STM32W108 at the heart of the design. This chip
serves as the transceiver and core microcontroller for the device. All
supporting (passive) components have been placed as in the reference design, so
as to minimize the potential for error. 

The schematic contains two SMD pushbuttons (SW1 and SW2), but these were removed
as they occupy too much space on the PCB. Likewise, the RP-SMA connector (P2) to
the left of the schematic is absent from the PCB for the same reason. In the
second version of the PCB, the components have been rearranged to make space for
this connector.

The design offers an external UART connection (P1) on a 4-pin header with
\SI{2.54}{mm} spacing. The chip can be programmed through P3, a 4-pin
Micro-MaTch\textsuperscript{\texttrademark} connector.

\begin{figure}[htb]
  \begin{center}
    \includegraphics[width=0.9\textwidth]{gfx/node-v1}
  \end{center}
  \caption{Version 1 PCB Prints}
  \label{fig:v1-prints}
\end{figure}

The PCB incorporates an RGB LED for debug purposes, but the footprint was found
to be incorrect. This has been corrected in the second version.

Version 1 of the design incorporates an audio amplification section, which was
later removed. This section was meant to amplify up to three chanels of analog
audio, as well as route one channel through a comparator, to wake the chip when
an incoming signal was detected. As this was not directly required in the
context of this project, it was scrapped, in favour of later designing a plug-in
module which would make use of the expansion port. The I/O port retains the pins
needed for such an application, namely the three analog inputs, as well as three
input capture interfaces.


\subsection{Antenna Design}

Radio frequency engineering, especially antenna design, is an area which
requires a lot of expertise. The design of an antenna itself would be
a sufficient topic for an entire Masters thesis by someone with experience in
that area. In this case, neither the author nor his supervisor have extensive
experience with antenna design. Attempts to get help from specialized labs
within EPFL were unsuccessful. In light of these facts, the information
contained in this section should be considered from the point of view of amateur
designers, seeking a way to use an antenna effectively, without any formal
background in that topic. Data and observations made here are mostly qualitative
and empirical.

\begin{figure}[htb]
  \begin{center}
    \includegraphics[width=0.6\textwidth]{gfx/snippets/altium-antenna}
  \end{center}
  \caption{PCB Trace Antenna}
  \label{fig:trace-antenna}
\end{figure}

Considering all this, we set out to design a prototype using whatever data was
available and accessible to us. Most decisions were made according to the
manufacturer's application notes, namely ``\emph{PCB design guidelines for the
STM32W108 platform}''\citep{AN3206} as well as ``\emph{Low cost PCB antenna for
\SI{2.4}{Ghz} radio : meander design}''\citep{AN3359}. Additional advice on
design and testing was gathered from ``\emph{The Dropout's Guide to PCB Trace
Antenna Design}''\citep{DropoutGuide}. 

What follows is a brief description of the various types of antenna one might
encounter, and a comparison between those designs. 

%\begin{figure}[bth]
%  \myfloatalign
%  \subfloat[PCB Trace Antenna]
%  {\includegraphics[width=.45\linewidth]{gfx/antennas/trace}}
%  \subfloat[Wire Antenna]
%  {\includegraphics[width=.45\linewidth]{gfx/antennas/wire}} \\
%  \subfloat[Chip Antenna]
%  {\includegraphics[width=.45\linewidth]{gfx/antennas/chip}}
%  \subfloat[``Ducky'' antenna]
%  {\includegraphics[width=.45\linewidth]{gfx/antennas/ducky}}
%  \caption{Various types of antennas}
%  \label{fig:antennas}
%\end{figure}

\subsubsection{Chip Antenna}

% TODO : include a photo of the starter kit to compare

A chip antenna requires little board space, usually less than a PCB trace
antenna. However, they are difficult to tune if needed, and increase the cost of
the overall bill of materials.

\begin{figure}[bth]
  \begin{center}
    \includegraphics[width=.45\linewidth]{gfx/antennas/chip}
  \end{center}
  \caption{Device with a chip antenna}
  \label{fig:chip-antenna}
\end{figure}
\marginpar{TODO : replace these pictures with photos of my own devices}

The example kit (\emph{STM32W-RFCKIT}) contains both a board with a chip antenna
and a PCB trace antenna, so it will be possible to make some comparisons between
the two options. In testing, covered in \autoref{ch:performance}, our current
prototype will be compared to another student's design, using a chip antenna
operating at the same frequency.

In the end, this option was discarded because of its cost and difficulty to
tune, if the need were to arise.

\subsubsection{Wire Antenna}

Wire antennas can prove quite efficient in transmission, but require a lot of
space around the board. They are also very sensitive to the length of wire
used, as well as the type of material. 

\begin{figure}[bth]
  \begin{center}
    \includegraphics[width=.45\linewidth]{gfx/antennas/wire}
  \end{center}
  \caption{Device with a wire antenna}
  \label{fig:wire-antenna}
\end{figure}

The most economical option is a 1/4 wavelength piece of wire\citep{slyt296}. At
\SI{2.45}{GHz}, this translates to
\begin{equation*}
  \frac{1}{4} \cdot \frac{c}{\SI{2.45}{GHz}} = \SI{3.06}{cm}.
  \label{eq:wavelength}
\end{equation*}

Clearly, this length of wire does not increase the cost of the board by any
significant margin, but it could render the enclosure design problematic.

\subsubsection{External Antenna}

The external antenna offers much better performance than the other antennas
mentioned in this listing. It does however require a lot more space than the
other solutions. Its cost is prohibitive in the current case : the cost of an
external antenna is almost the same as all other components of the design put
together.

\begin{figure}[bth]
  \begin{center}
    \includegraphics[width=.45\linewidth]{gfx/antennas/ducky}
  \end{center}
  \caption{An external ``rubber ducky'' antenna}
  \label{fig:ducky-antenna}
\end{figure}

However, we recognize that it may be desirable for some nodes to have better
performance than others, especially in the case of router nodes. Therefore,
version 2 of the transceiver incorporates a footprint for an external antenna on
an RP-SMA connector.

\subsubsection{PCB Antenna}

\begin{figure}[bth]
  \begin{center}
    \includegraphics[width=.45\linewidth]{gfx/antennas/trace}
  \end{center}
  \caption{Example device with a PCB trace antenna}
  \label{fig:trace-antenna}
\end{figure}

For reasons similar to those expressed in the aforementioned guide, the
prototype device will use a PCB trace antenna. In addition to that, we will
include a footprint for adding an RP-SMA connector. The reasons for this choice
are as follows:

\begin{itemize}
  \item It is the cheapest option, as it requires no external components. Its
    only cost is board area.
  \item It can be easily combined with an RP-SMA connector. A capacitor can be
    soldered in one of two places to determine which antenna will be used. 
  \item A reference design is easily available, as well as a starter kit which
    can be used as a benchmark for performance tests.
  \item Results are generally more repeatable than with a wire antenna, and less
    dependent on solder joints than a chip antenna.
  \item Occupies less space than a wire antenna, therefore easier to fit inside
    an enclosure.
\end{itemize}

\subsubsection{Summary}

In summary, there is no ``ideal'' choice for an antenna in a wireless project.
Each model described previously has its own advantages and disadvantages. The
choice will mostly be a compromise between performance, price and board space.
\autoref{tab:antennas}, taken from a Texas Instruments application
note\citep{slyt296}, does a decent job of summing up the differences between
different types of antennas, for quick reference.

Note that in this table, the antennas which we refer to here as ``wire antenna''
and ``external antenna'' are grouped under one common category, as ``whip
antennas''.

\begin{table}[tbh]
    \myfloatalign
    \makebox[\textwidth][c]{
    \begin{tabularx}{1.3\textwidth}{l|p{0.45\textwidth}|p{0.45\textwidth}} \toprule
    \tableheadline{Antenna Types}
    & \tableheadline{Pros}
    & \tableheadline{Cons} \\ \midrule
    PCB Antenna & 
    Low cost

    Good performance is possible

    Small size is possible at high frequencies &
    Difficult to design small and efficient antennas

    Potentially large size at low frequencies \\
    \midrule
    Chip Antenna  &
    Small size  &
    Medium performance

    Medium cost \\
    \midrule
    Whip Antenna  &
    Good performance  &
    High cost

    Difficult to fit in many applications \\
    \bottomrule
  \end{tabularx}
  }
  \caption[Antenna Comparison]{Antenna Comparison}
  \label{tab:antennas}
\end{table}

% TODO 
Explain some details of the design, explain choice for components, say a few
words about the audio part?

We used application note about PCB design : \cite{AN3206} as well as
\cite{AN3359} for antenna design

Note some confusion about specifications : ``\emph{The STM32W108 Reference
Design is built on a 4-layer, FR-4 epoxy PCB}''\citep{AN3206}. However, the
Gerber files available for download from the website as well as the directives
given in AN3359\cite{AN3359} suggest using a 2-layer PCB.

Used advice from \cite{DropoutGuide} since we couldn't get any help from EPFL
experts.


\section{Version 2}

% TODO 
Explain why we dropped the audio part.
% TODO 
Complete reasoning for each pin of the expansion port.
% TODO 
Why we dropped the programming connector.

\begin{figure}[htb]
  \begin{center}
    \includegraphics[width=0.9\textwidth]{gfx/node-v2}
  \end{center}
  \caption{Version 2 PCB Prints}
  \label{fig:v2-prints}
\end{figure}

\subsection{I/O Connector}\label{sub:io-connector}

In this revision, the emphasis was placed on minimizing board space. In order to
do this, all the connectors were removed, and replaced by a small-pitch general
purpose connector, which should be suitable for most needs.

Connectivity is provided by an 18-pin connector in \SI{2}{mm} pitch. It provides
power, drawn from the \SI{3.3}{V} regulator on the board. 

\begin{table}
    \myfloatalign
  \begin{tabularx}{\textwidth}{llX} \toprule
    \tableheadline{Pin \#} & \tableheadline{Port Number}
    & \tableheadline{Function} \\ \midrule
    1   & +3.3V   & Power from regulator    \\
    2   & GND     & Ground Connection       \\
    3   & PC4     & SWDIO : Debug Data      \\
    \midrule
    4   & SWCLK   & SWCLK : Debug Clock     \\
    5   & PA0     & PWM4        \\
    6   & PB0     & GPIO 3    \\
    \midrule
    7   & PA6     & Timer1 Channel 3        \\
    8   & PA7     & Timer1 Channel 4     \\
    9   & PB3     & GPIO 1 / CTS    \\

%    \bottomrule
%  \end{tabularx}
%  \caption[I/O Connector]{I/O Connector}
%  \label{tab:io-connector}
%\end{table}
%
%\begin{table}
%    \myfloatalign
%  \begin{tabularx}{\textwidth}{llX} \toprule
%    \tableheadline{Pin \#} & \tableheadline{Port Number}
%    & \tableheadline{Function} \\ \midrule
    \midrule
    10   & PB4     & GPIO 2 / RTS        \\
    11   & PB1     & UART TX     \\
    12   & PB2     & UART RX     \\
    \midrule
    13   & PB7     & ADC2        \\
    14   & PA1     & PWM2     \\
    15   & PB6     & ADC1     \\
    \midrule
    16   & PA2     & PWM1        \\
    17   & PB5     & ADC0     \\
    18   & PA3     & PWM0     \\
    \bottomrule
  \end{tabularx}
  \caption[I/O Connector]{I/O Connector}
  \label{tab:io-connector}
\end{table}

Each pin's function is summarized in \autoref{tab:io-connector}. The detailed
functionality is as follows:

\begin{itemize}
  \item \emph{GPIO} : General Purpose I/O pins. These pins have no particular
    special function. They will most likely be used to read switches or activate
    leds and relays.
    
    GPIO1 and GPIO2 can be used respetively as CTS and RTS lines if the UART
    module is to be used in flow control mode.
  \item \emph{Timer 1} : Two channels of timer 1 are provided. These can be used
    as output compare channels for \ac{PWM} or as input captures.
  \item \emph{Timer 2} : All four channels of timer 2 are broken out to the
    expansion port. This allows the user to perform an input capture on all four
    channels of the same timer, for instance to measure the distance between
    peaks on separate channels with a high resolution. These channels can also
    be used as \ac{PWM} outputs.

    Furthermore, pins 14 and 16 can be used for $I^{2}C$. All four timer 2 pins
    can serve the function of an SPI controller.
  \item \emph{Debug} : The device can be programmed and debugged using an
    \emph{ST-Link} device or by JTAG. The \emph{ST-Link} was chosen for this
    design as it requires only two pins to function. Connecting the board to the
    debugger is covered in \autoref{sub:debugging}.
    % TODO : make a section about programming, debugging, the connector etc
  \item \emph{<++>} : <++>
\end{itemize}

Naturally, each of these pins can also be used as general purpose inputs or
outputs.

\subsection{Debug LED}

An RGB LED is available on port C for debug purposes. Connections are described
in \autoref{tab:led-assignments}. The power consumption of each colour was
measured with the given resistor values, for a \SI{3.3}{V} operating voltage.
These results were obtained by measuring the total consumption of the device
with the LED on, then subtracting the consumption with the LED turned off.

\begin{table}[tbh]
    \myfloatalign
  \begin{tabularx}{\textwidth}{llX} \toprule
    \tableheadline{Pin Name} & \tableheadline{LED Colour}
    & \tableheadline{Measured Power Consumption} \\ \midrule
    PC0   & Red     & \SI{8.7}{mA}   \\
    PC1   & Green   & \SI{3.9}{mA}   \\
    PC2   & Blue    & \SI{4.6}{mA}   \\
    \bottomrule
  \end{tabularx}
  \caption[LED Assignment]{LED Assignment}
  \label{tab:led-assignments}
\end{table}

No effort was made to reduce power consumption of the LED, since its purpose is
for debug only. If it is required as an indicator, we recommend using very short
pulses to reduce the average power consumption of the indicator. If necessary,
the resistor values can also be adjusted.

% TODO : measure consumption per colour and describe the procedure.
% TODO : mention the joule losses in the resistors?

\subsection{Debugging}\label{sub:debugging}

Programming and debugging of the platform can be done using an \emph{ST-Link}
device. There are two options for this : either by purchasing a standalone
device through standard distributors such as Digi-Key, or by using an \emph{STM
Discovery Kit}\footnote{These can be obtained quite easily for less than CHF~12.
See \url{http://www.st.com/stm32-discovery} for details.}

\begin{figure}[bth]
  \myfloatalign
  \subfloat[Version 1 Debug Link]
  {\includegraphics[width=.45\linewidth]{gfx/placeholder-200x200}}
  \subfloat[Version 2 Debug Link]
  {\includegraphics[width=.45\linewidth]{gfx/placeholder-200x200}}
  \caption{ST-Link Debugger Connections}
  \label{fig:st-link}
\end{figure}

\autoref{tab:programming-connector} describes the connections between the
discovery kit and the development platform. Pin number one on the platform
refers to the upper left hand corner of the 18-pin connector, while pin one on
the discovery kit is the upper-most pin, closest to the USB connector. Both
versions of the connectors are illustrated in \autoref{fig:st-link}.


\begin{table}[tbh]
    \myfloatalign
  \begin{tabularx}{\textwidth}{XXXX} \toprule
    \tableheadline{SWD Pin \#} & \tableheadline{Cable}
    & \tableheadline{Pin function} & \tableheadline{Platform Pin \#}\\ \midrule
    1   & Red      & Vcc    & 2 \\
    2   & Yellow   & SWCLK  & 3 \\
    3   & Black    & Gnd    & 1 \\
    4   & Green    & SWDIO  & 4 \\
    \bottomrule
  \end{tabularx}
  \caption[Programming connector on ST-Link]{Programming connector on ST-Link}
  \label{tab:programming-connector}
\end{table}

Programmers are usually supported by whichever programming environment is being
used. In order to test the examples that are provided along with the files
related to this project, the simplest solution is to use ST's own
utility.\footnote{The ST-Link Utility can be downloaded from
\url{http://www.st.com/internet/evalboard/product/219866.jsp}.}



\section{Possible future improvements}

If the goal is to minimize board size, there are a few straightforward
modifications that can still be made. The current version is a compromise
between versatility and compactness.

\begin{itemize}
  \item \emph{I/O Connector} : The current connector was chosen to offer a large
    choice of inputs and outputs for various applications. If the target
    application only requires a small number of inputs and outputs, a much
    smaller connector can be used. It would also be feasible to use
    a \SI{1.27}{mm} pitch connector. 
    Furthermore, for a production run, it would be advisable to separate the
    debug and programming connections (SWDIO and SWCLK) from the expansion port.
    These could be replaced either by a dedicated connector or by pads to be
    used in a spring-loaded jig.
  \item \emph{Debug LED} : This is provided for convenience during programming.
    It is obviously not required for most applications. 
  \item \emph{\SI{3.3}{V} Regulator} : This is another item that was included
    for convenience during development. If the target application is capable of
    providing a \SI{3}{V} output, the regulator on this board can be omitted.
\end{itemize}

\section{Example application : Triac module}

The main goal of this project was not to produce a suite of designs with various
functions, but rather to come up with a general-purpose platform which could
later be extended with various add-on modules. The expansion port should provide
plenty of options to connect boards with sensors, relays and switches.

Despite this, one small example module was developed to use as an example. The
triac module is a simple board that can be used to control AC loads, up to
\SI{4}{A} at \SI{230}{V}.

\begin{figure}[htb]
  \begin{center}
    \includegraphics[width=0.9\textwidth]{gfx/TriacModule-Prints}
  \end{center}
  \caption{Triac Module}
  \label{fig:triac-module}
\end{figure}

This module is one of the simplest applications of a triac\footnote{a TRIAC,
short for ``Triode for Alternating Current'', is an electronic component which
allows current through its terminals in both directions when its gate is
triggered by even a low-voltage source. Its use is very similar to that of
a relay, with the difference that it is solid-state.}
: it contains an opto-triac, which drives the triac from a low-voltage DC
source. Connector P3, on the left-hand side, can be connected to
a microcontroller output. When \SI{3}{V} are applied to it, the triac will allow
current to flow through the right hand side of the circuit. The mains power and
load are to be connected respectively to P1 and P2. A standard \SI{5x20}{mm}
fuse offers a simple protection against short circuit.

It is important to be cautious when working with mains power. It is best to use
a decoupling transformer, to separate oneself from the live mains, as well as
cover up the exposed circuitry with non-conductive materials, such as
heat-resistant adhesive tape.

\begin{figure}[htb]
  \begin{center}
    \includegraphics[width=1.2\textwidth]{gfx/Triac-Schematic}
  \end{center}
  \caption{Triac Module Schematic}
  \label{fig:triac-schematic}
\end{figure}

\section{Cost}\label{sec:cost}

\autoref{tab:bom-v2} contains the detailed bill of materials for the latest PCB
version. It can be found in \autoref{ch:appendix-design} under
\autoref{sub:bom-v2}.

From this table, we can see that in reasonably large quantities (1000 pieces),
we have reached our target of providing wireless functionality for less than
10~CHF.\footnote{as stated in \autoref{sec:cdc}.}


\begin{table}
    \myfloatalign
  \begin{tabularx}{0.4\textwidth}{ll} \toprule
    \tableheadline{Components} & \tableheadline{Part} \\
        \midrule
        Microcontroller & 57.07\% \\ 
        LDO             & 20.00\% \\ 
        Crystal         & 5.68\%  \\ 
        LED             & 4.05\%  \\ 
        RF Filter       & 2.05\%  \\ 
        Inductors       & 1.16\%  \\ 
        Capacitors      & 0.60\%  \\ 
        Balun           & 0.22\%  \\ 
        JST Header      & 0.04\%  \\ 
        Resistors       & 0.01\%  \\
  \end{tabularx}
  \caption[Component price breakdown]{Component price breakdown}
  \label{tab:price-breakdown}
\end{table}

We can further break down the costs of the device: \autoref{tab:price-breakdown}
shows the distribution of costs per type of component. As we can see, the
microcontroller accounts for over half the cost of all components.

Currently, total component cost adds up to 5.69~CHF per item. This is only the
raw cost of components from the supplier and does not include PCB fabrication or
assembly.

If we wish only to use this design to add wireless connectivity to an existing
project, we may not need the indicator LED or the \SI{3.3}{V} voltage regulator.
In that case, we may remove them from the design, thus saving 24\% of the total
component cost. The resulting cost is therefore reduced to 4.27~CHF.
