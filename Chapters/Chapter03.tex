%*****************************************
\chapter{Hardware development}\label{ch:examples}
%*****************************************
%\setcounter{figure}{10}

The hardware for this device went through two major revisions. The first was
destined to assess the functionality of the chip, and make sure that it was
possible to establish communication between two nodes. Since design of a system
with a built-in antenna and transceiver was new to the author, it was essential
to proceed in small, manageable steps to help debugging if problems were to
arise. For these same reasons, the current designs remain very close to the
suggested reference designs.

After testing several versions and multiple components, and comparing them to
existing designs with other chips, we have formed some ideas on ways to improve
the design, which we will cover in [blah] % TODO : reference the chapter
Indeed, although we have obtained performance very similar to the reference
design provided by ST Microelectronics, we believe it is possible to achieve
a better design than the reference

[Image] compare both versions side by side. Show the dimensions on each.

\section{Version 1}

Version 1 is comprised of the STM32W108 at the heart of the design. This chip
serves as the transceiver and core microcontroller for the device. All
supporting (passive) components have been placed as in the reference design, so
as to minimize the potential for error. 

The schematic contains two SMD pushbuttons (SW1 and SW2), but these were removed
as they occupy too much space on the PCB. Likewise, the RP-SMA connector (P2) to
the left of the schematic is absent from the PCB for the same reason. In the
second version of the PCB, the components have been rearranged to make space for
this connector.

The design offers an external UART connection (P1) on a 4-pin header with
\SI{2.54}{mm} spacing. The chip can be programmed through P3, a 4-pin
Micro-MaTch\textsuperscript{\texttrademark} connector.

The PCB incorporates an RGB LED for debug purposes, but the footprint was found
to be incorrect. This has been corrected in the second version.

The difference that the reader will most likely notice first between the two
revisions of the PCB is the removal of the audio amplification in the second
version. 

% TODO : explain why we had this, and why we removed it.


\subsection{Antenna Design}

% TODO : add references
% Antenna design STM
% AN3206 PCB design guidelines for the STM32W108 platform

We used application note about PCB design : \cite{AN3206} as well as
\cite{AN3359} for antenna design

Note some confusion about specifications : \emph{``The STM32W108 Reference
Design is built on a 4-layer, FR-4 epoxy PCB.''}\cite{AN3206}. However, the
Gerber files available for download from the website as well as the directives
given in AN3359\cite{AN3359} suggest using a 2-layer PCB.

Used advice from \cite{DropoutGuide} since we couldn't get any help from EPFL
experts.

\subsubsection{Chip Antenna}

% TODO : include a photo of the starter kit to compare

\subsubsection{Wire Antenna}
\subsubsection{External Antenna}
\subsubsection{PCB Antenna}

Explain some details of the design, explain choice for components, say a few
words about the audio part?

\section{Version 2}

Explain why we dropped the audio part. Complete reasoning for each pin of the
expansion port. Why we dropped the programming connector.

\subsection{I/O Connector}

In this revision, the emphasis was placed on minimizing board space. In order to
do this, all the connectors were removed, and replaced by a small-pitch general
purpose connector, which should be suitable for most needs.

Connectivity is provided by an 18-pin connector in \SI{2}{mm} pitch. It provides
power, drawn from the \SI{3.3}{V} regulator on the board. 

\begin{table}
    \myfloatalign
  \begin{tabularx}{\textwidth}{llX} \toprule
    \tableheadline{Pin \#} & \tableheadline{Port Number}
    & \tableheadline{Function} \\ \midrule
    1   & +3.3V   & Power from regulator    \\
    2   & GND     & Ground Connection       \\
    3   & PC4     & SWDIO : Debug Data      \\
    \midrule
    4   & SWCLK   & SWCLK : Debug Clock     \\
    5   & PA0     & PWM4        \\
    6   & PB0     & GPIO 3    \\
    \midrule
    7   & PA6     & Timer1 Channel 3        \\
    8   & PA7     & Timer1 Channel 4     \\
    9   & PB3     & GPIO 1 / CTS    \\

%    \bottomrule
%  \end{tabularx}
%  \caption[I/O Connector]{I/O Connector}
%  \label{tab:io-connector}
%\end{table}
%
%\begin{table}
%    \myfloatalign
%  \begin{tabularx}{\textwidth}{llX} \toprule
%    \tableheadline{Pin \#} & \tableheadline{Port Number}
%    & \tableheadline{Function} \\ \midrule
    \midrule
    10   & PB4     & GPIO 2 / RTS        \\
    11   & PB1     & UART TX     \\
    12   & PB2     & UART RX     \\
    \midrule
    13   & PB7     & ADC2        \\
    14   & PA1     & PWM2     \\
    15   & PB6     & ADC1     \\
    \midrule
    16   & PA2     & PWM1        \\
    17   & PB5     & ADC0     \\
    18   & PA3     & PWM0     \\
    \bottomrule
  \end{tabularx}
  \caption[I/O Connector]{I/O Connector}
  \label{tab:io-connector}
\end{table}

Each pin's function is summarized in \autoref{tab:io-connector}. The detailed
functionality is as follows:

\begin{itemize}
  \item \emph{GPIO} : General Purpose I/O pins. These pins have no particular
    special function. They will most likely be used to read switches or activate
    leds and relays.
    
    GPIO1 and GPIO2 can be used respetively as CTS and RTS lines if the UART
    module is to be used in flow control mode.
  \item \emph{Timer 1} : Two channels of timer 1 are provided. These can be used
    as output compare channels for \ac{PWM} or as input captures.
  \item \emph{Timer 2} : All four channels of timer 2 are broken out to the
    expansion port. This allows the user to perform an input capture on all four
    channels of the same timer, for instance to measure the distance between
    peaks on separate channels with a high resolution. These channels can also
    be used as \ac{PWM} outputs.

    Furthermore, pins 14 and 16 can be used for $I^{2}C$. All four timer 2 pins
    can serve the function of an SPI controller.
  \item \emph{<++>} : <++>
\end{itemize}

Naturally, each of these pins can also be used as general purpose inputs or
outputs.

\subsection{Debug LED}

An RGB LED is available on port C for debug purposes. Connections are described
in \autoref{tab:led-assignments}. The power consumption of each colour was
measured with the given resistor values, for a \SI{3.3}{V} operating voltage.

No effort was made to reduce power consumption of the LED, since its purpose is
for debug only. If it is required as an indicator, we recommend using very short
pulses to reduce the average power consumption of the indicator. If necessary,
the resistor values can also be adjusted.

% TODO : measure consumption per colour and describe the procedure.
% TODO : mention the joule losses in the resistors?

\begin{table}
    \myfloatalign
  \begin{tabularx}{\textwidth}{llX} \toprule
    \tableheadline{Pin Name} & \tableheadline{LED Colour}
    & \tableheadline{Measured Power Consumption} \\ \midrule
    PC0   & Red     & \SI{20}{mA}   \\
    PC1   & Green   & \SI{20}{mA}   \\
    PC2   & Blue    & \SI{20}{mA}   \\
    \bottomrule
  \end{tabularx}
  \caption[LED Assignment]{LED Assignment}
  \label{tab:led-assignments}
\end{table}



\section{Possible future improvements}

If the goal is to minimize board size, there are a few straightforward
modifications that can still be made. The current version is a compromise
between versatility and compactness.

\begin{itemize}
  \item \emph{I/O Connector} : The current connector was chosen to offer a large
    choice of inputs and outputs for various applications. If the target
    application only requires a small number of inputs and outputs, a much
    smaller connector can be used. It would also be feasible to use
    a \SI{1.27}{mm} pitch connector. 
    Furthermore, for a production run, it would be advisable to separate the
    debug and programming connections (SWDIO and SWCLK) from the expansion port.
    These could be replaced either by a dedicated connector or by pads to be
    used in a spring-loaded jig.
  \item \emph{Debug LED} : This is provided for convenience during programming.
    It is obviously not required for most applications. 
  \item \emph{\SI{3.3}{V} Regulator} : This is another item that was included
    for convenience during development. If the target application is capable of
    providing a \SI{3}{V} output, the regulator on this board can be omitted.
\end{itemize}

\section{Example application : Triac module}

Explain that the aim of the project was not to produce a design with lots of
applications but rather to provide a test platform and sufficient data to
compare available solutions. Application-specific modules are to be developed
later on when an application has been chosen.


