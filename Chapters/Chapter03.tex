%*****************************************
\chapter{Hardware development}\label{ch:examples}
%*****************************************
%\setcounter{figure}{10}

The hardware for this device went through two major revisions. The first was
destined to assess the functionality of the chip, and make sure that it was
possible to establish communication between two nodes. Since design of a system
with a built-in antenna and transceiver was new to the author, it was essential
to proceed in small, manageable steps to help debugging if problems were to
arise. For these same reasons, the current designs remain very close to the
suggested reference designs.

After testing several versions and multiple components, and comparing them to
existing designs with other chips, we have formed some ideas on ways to improve
the design, which we will cover in [blah] % TODO : reference the chapter
Indeed, although we have obtained performance very similar to the reference
design provided by ST Microelectronics, we believe it is possible to achieve
a better design than the reference

[Image] compare both versions side by side. Show the dimensions on each.

\section{Version 1}

Version 1 is comprised of the STM32W108 at the heart of the design. This chip
serves as the transceiver and core microcontroller for the device. All
supporting (passive) components have been placed as in the reference design, so
as to minimize the potential for error. 

The schematic contains two SMD pushbuttons (SW1 and SW2), but these were removed
as they occupy too much space on the PCB. Likewise, the RP-SMA connector (P2) to
the left of the schematic is absent from the PCB for the same reason. In the
second version of the PCB, the components have been rearranged to make space for
this connector.

The design offers an external UART connection (P1) on a 4-pin header with
\SI{2.54}{mm} spacing. The chip can be programmed through P3, a 4-pin
Micro-MaTch\textsuperscript{\texttrademark} connector.

The PCB incorporates an RGB LED for debug purposes, but the footprint was found
to be incorrect. This has been corrected in the second version.

The difference that the reader will most likely notice first between the two
revisions of the PCB is the removal of the audio amplification in the second
version. 

% TODO : explain why we had this, and why we removed it.


\subsection{Antenna Design}

% TODO : add references
% Antenna design STM
% AN3206 PCB design guidelines for the STM32W108 platform

We used application note about PCB design : \cite{AN3206} as well as
\cite{AN3359} for antenna design

Note some confusion about specifications : \emph{``The STM32W108 Reference
Design is built on a 4-layer, FR-4 epoxy PCB.''}\cite{AN3206}. However, the
Gerber files available for download from the website as well as the directives
given in AN3359\cite{AN3359} suggest using a 2-layer PCB.

Used advice from \cite{DropoutGuide} since we couldn't get any help from EPFL
experts.

\subsubsection{Chip Antenna}

% TODO : include a photo of the starter kit to compare

\subsubsection{Wire Antenna}
\subsubsection{External Antenna}
\subsubsection{PCB Antenna}

Explain some details of the design, explain choice for components, say a few
words about the audio part?

\section{Version 2}

Explain why we dropped the audio part. Complete reasoning for each pin of the
expansion port. Why we dropped the programming connector.

\subsection{I/O Connector}

\begin{table}
    \myfloatalign
  \begin{tabularx}{\textwidth}{llX} \toprule
    \tableheadline{Pin \#} & \tableheadline{Port Number}
    & \tableheadline{Function} \\ \midrule
    1   & +3.3V   & Power from regulator    \\
    2   & GND     & Ground Connection       \\
    3   & PC4     & SWDIO : Debug Data      \\
    \midrule
    4   & SWCLK   & SWCLK : Debug Clock     \\
    5   & PA0     & Timer2 Channel 1        \\
    6   & PB0     & General Purpose I/O     \\
    \midrule
    7   & PX0     & Derp, this will be quite a long explanation, it might
                    even span several lines \\
    8   & PX0     & Herp     \\
    9   & PX0     & Durr     \\
    \midrule
    10   & PX0     & Derp        \\
    11   & PX0     & Herp     \\
    12   & PX0     & Durr     \\
    \midrule
    13   & PX0     & Derp        \\
    14   & PX0     & Herp     \\
    15   & PX0     & Durr     \\
    \midrule
    16   & PX0     & Derp        \\
    17   & PX0     & Herp     \\
    18   & PX0     & Durr     \\
    \bottomrule
  \end{tabularx}
  \caption[Frequency comparison]{Frequency comparison}
  \label{tab:frequency-comparison}
\end{table}

\section{Example application : Triac module}

Explain that the aim of the project was not to produce a design with lots of
applications but rather to provide a test platform and sufficient data to
compare available solutions. Application-specific modules are to be developed
later on when an application has been chosen.


