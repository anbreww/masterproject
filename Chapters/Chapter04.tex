%************************************************
\chapter{Performance tests}\label{ch:performance} % $\mathbb{ZNR}$
%************************************************

In this chapter, we describe what we want to test for, how we tested it, what we
compared it to, and what the results were. Then a discussion of the results.

\section{Parameters to test for}

\begin{itemize}
  \item \emph{Power Consumption} : a module must be able to survive on minimal
    power. To determine whether this is possible, we have two concurrent
    approaches : measure the minimum possible current consumption (in a sleep
    state with a wake timer, as well as in deep sleep with a wake interrupt),
    measure the current during a transmission and establish the connection
    between transmission duty cycle and average current consumption. Using data
    gathered from a month-long test using and indoor energy harvesting
    datalogger, we can deduce the maximum duty cycle that can sustain the
    device, which will then allow us to determine which application are suitable
    for this kind of power source.
  \item \emph{RF Power and Range} : Each revision of the PCB will be tested in
    typical working conditions for transmission range and compared with the
    reference designs. We will also take a measurement of the power received at
    a given distance when transmitting a sine wave at various power settings and
    frequencies.
  \item \emph{RF Components} : Some components have a very high influence on the
    total cost of the device. We shall test alternatives to the recommended
    components to determine if we can lower the cost without compromising
    performance.
  \item \emph{<++>} : <++>
\end{itemize}

\section{Tests}

Here we'll describe the tests and show the results

\subsection{Current consumption}

Show the measured current consumption in various modes, show the associated code
to achieve that state, and compare with values from the datasheet.

\begin{lstlisting}[language=C,caption=Example code for sleep mode A]
  int main(void)
  {
    STlibSleep(SLEEP_MODE_BLAH);
    sleep();
  }
\end{lstlisting}

\subsubsection{Duty Cycle}

Taking into account the time it takes to wake the RF part (take from datasheet
since our scopes suck), derive a formula for the average consumption.

\subsubsection{Energy Harvesting}

Show a few plots of the harvesting results, give some average values for various
situations, and give some duty cycle estimates.

From this, give some concrete examples (i.e. for reading a temperature, it would
take n microseconds to wake the device, take a reading, m millis to transmit,
etc, since duty cycle is D, we can expect to transmit a temperature reading once
every M minutes.)

Propose some alternatives in case we want a higher sample rate : wake the device
to a slow mode, without RF, log to EEPROM and then send 10 readings at a time.

\subsection{RF Power}


\begin{figure}[htb]
  \begin{center}
    \includegraphics[width=1.0\textwidth]{gfx/snippets/spectrum}
  \end{center}
  \caption{802.15.4 Channel Allocation\citep{hunn2010}}
  \label{fig:channel-allocation}
\end{figure}

At \SI{2.4}{GHz}, the spectrum for the IEEE 802.15.4 is divided into 16 equal
channels, each \SI{2}{MHz} wide, spaced \SI{5}{MHz} apart. The spectrum is
centered around \SI{2.45}{GHz}.\citep[pg. 29]{ieee802154}

Channel frequency is defined as $Fc = 2405 + 5 (k-11)$ in Megaherz, for $k
= 11..26$.

Illustrate the test setup, clearly indicate measuring distance. Test for various
orientations.

\subsection{RF Range}

Take possibly the best performing device from the previous tests (hopefully the
one with the ducky antenna), and then see how far I can get with the test
modules in various situations. Perhaps test in a corridor with a double
decameter.

\subsection{RF Components}

List recommended components and their BOM price. Compare those with some lower
priced examples (and compare their datasheet specs, like insertion loss). Then
replace those in a design and then run a power test on them, in similar
operating conditions.




\section{Comparison Tests}

In parallel to this project, it was discovered that a similar student project
was taking place. We decided to set up a simple experiment to compare, somewhat
informally, the performance of both our designs.

In this test, illustrated in \autoref{fig:test-setup}, we measure the
transmission of the antenna, using an SMA connector soldered through the board,
as described in % TODO : add ref to dropout's guide.

\begin{figure}[htb]
  \begin{center}
    \makebox[\textwidth][c]{
    \includegraphics[width=1.5\textwidth]{gfx/tests/test02-setup-cropped}
    }
  \end{center}
  \caption{Comparison Test Setup}
  \label{fig:test-setup}
\end{figure}

The network analyzer sweeps a range of frequencies, spanning \SI{0.5}{GHz}
around \SI{2.45}{GHz} and measures the energy received through a ``rubber
ducky'' monopole antenna. Each antenna was tested in two different orientations.
First, the transceiver is placed approximately \SI{50}{cm} away from the
antenna. The antenna is vertical, as in \autoref{fig:test-setup}, secured to
a non-metallic support. The device under test is then placed at the other end of
the setup, its feedline against the support. In this orientation, the plane
containing the PCB is normal to the monopole antenna.

In the second test, the device is placed at the same distance, turned 90° so
that the antenna is now ``above'' the device. Each test was done in the same way
for both devices under test.

The device developed in this project will be referred to as \textbf{``device A''}.
Jean-Baptiste's device will be referred to as \textbf{``device B''}.


The following screenshots show the results of our four tests.

\begin{figure}[h!]
  \begin{center}
    \includegraphics[width=0.8\textwidth]{gfx/tests/test01-result}
  \end{center}
  \caption{Test 1 : device A, normal}
  \label{fig:test01-result}
\end{figure}

\begin{figure}[h!]
  \begin{center}
    \includegraphics[width=0.8\textwidth]{gfx/tests/test02-result}
  \end{center}
  \caption{Test 2 : device A, parallel}
  \label{fig:test02-result}
\end{figure}

\begin{figure}[h!]
  \begin{center}
    \includegraphics[width=0.8\textwidth]{gfx/tests/test03-result}
  \end{center}
  \caption{Test 3 : device B, normal}
  \label{fig:test03-result}
\end{figure}

\begin{figure}[h!]
  \begin{center}
    \includegraphics[width=0.8\textwidth]{gfx/tests/test04-result}
  \end{center}
  \caption{Test 4 : device B, parallel}
  \label{fig:test04-result}
\end{figure}

\pagebreak

The results are summarized in \autoref{tab:comparison-results}. This table shows
the attenuation measured in each case, in the center of the operating bandwidth
(\ie~\SI{2.45}{GHz})

It is worth noting at this point that these results are not to be taken for
their quantitative values. Such testing requires a strict methodology in
a controlled environment, which was not available in the context of this work.
Instead, we present these results simply as a matter of comparison between two
similar projects, using two different choices of antenna designs.

\begin{table}
    \myfloatalign
  \begin{tabularx}{0.6\textwidth}{Xcc} \toprule
    \tableheadline{} & \tableheadline{normal}
    & \tableheadline{parallel} \\ \midrule
    device A  & \SI{-63.3}{dB}  & \SI{-53.0}{dB} \\
    device B  & \SI{-51.1}{dB}  & \SI{-43.1}{dB} \\
    \bottomrule
  \end{tabularx}
  \caption[Comparison results]{Comparison results}
  \label{tab:comparison-results}
\end{table}

\subsection{Observations}

From this test, we can see that even though this project has achieved
performance similar to the reference design, it is possible to do better, given
the same requirements. It is quite clear from the approximately \SI{10}{dB}
difference between both designs, that we can expect better results from using
a different antenna in future versions.

It remains to be seen however if it is possible to improve the design using only
a PCB trace antenna, or if we must balance this potential performance
improvement with the cost of a chip antenna.

Although in theory these measurements were taken from the same point in the
signal path, our device uses a band-pass filter in order to attenuate noise
outside of the transmission band. This is often necessary to pass certification
tests, as there is a limit to the amounts of radiation that can be produced
outside of the legal frequency range. The band-pass filter has an insertion loss
of up to \SI{3}{dB}, which must be taken into account. Even so, device B still
outperforms our own.

Unfortunately, we do not know for a fact whether such a protection is necessary
or not, as it would require equipment and qualifications beyond our reach. 
% TODO : add that the potential gain seems to justify that sort of testing.
% probably note that if a project like this were to be taken any further, we'd
% need some specialists to work on the RF part. A semester project testing
% antenna designs at the lema would be nice

\begin{figure}[htb]
  \begin{center}
    \includegraphics[width=0.9\textwidth]{gfx/radiation-pattern}
  \end{center}
  \caption{PCB Antenna Radiation Pattern\citep{AN3359}}
  \label{fig:radiation-pattern}
\end{figure}

Our final observation is that both devices are very sensitive to changes in
orientation. Both antennas are meant to be omni-directional, however, some
directions are clearly favoured over others. If we refer to page 12 of the
antenna design guide\citep{AN3359}, we can see that the first orientation tested
places the receiving antenna in the worst possible location in the radiation
pattern. For convenience, this pattern is reproduced in
\autoref{fig:radiation-pattern}
