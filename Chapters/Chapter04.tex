%************************************************
\chapter{Performance tests}\label{ch:performance} % $\mathbb{ZNR}$
%************************************************

In this chapter, we describe what we want to test for, how we tested it, what we
compared it to, and what the results were. Then a discussion of the results.

\section{Parameters to test for}

\begin{itemize}
  \item \emph{Power Consumption} : a module must be able to survive on minimal
    power. To determine whether this is possible, we have two concurrent
    approaches : measure the minimum possible current consumption (in a sleep
    state with a wake timer, as well as in deep sleep with a wake interrupt),
    measure the current during a transmission and establish the connection
    between transmission duty cycle and average current consumption. Using data
    gathered from a month-long test using and indoor energy harvesting
    datalogger, we can deduce the maximum duty cycle that can sustain the
    device, which will then allow us to determine which application are suitable
    for this kind of power source.
  \item \emph{RF Power and Range} : Each revision of the PCB will be tested in
    typical working conditions for transmission range and compared with the
    reference designs. We will also take a measurement of the power received at
    a given distance when transmitting a sine wave at various power settings and
    frequencies.
  \item \emph{RF Components} : Some components have a very high influence on the
    total cost of the device. We shall test alternatives to the recommended
    components to determine if we can lower the cost without compromising
    performance.
  \item \emph{<++>} : <++>
\end{itemize}

\section{Tests}

Here we'll describe the tests and show the results

\subsection{Current consumption}

Show the measured current consumption in various modes, show the associated code
to achieve that state, and compare with values from the datasheet.

\begin{lstlisting}[language=C,caption=Example code for sleep mode A]
  int main(void)
  {
    STlibSleep(SLEEP_MODE_BLAH);
    sleep();
  }
\end{lstlisting}

\subsubsection{Duty Cycle}

Taking into account the time it takes to wake the RF part (take from datasheet
since our scopes suck), derive a formula for the average consumption.

\subsubsection{Energy Harvesting}

Show a few plots of the harvesting results, give some average values for various
situations, and give some duty cycle estimates.

From this, give some concrete examples (i.e. for reading a temperature, it would
take n microseconds to wake the device, take a reading, m millis to transmit,
etc, since duty cycle is D, we can expect to transmit a temperature reading once
every M minutes.)

Propose some alternatives in case we want a higher sample rate : wake the device
to a slow mode, without RF, log to EEPROM and then send 10 readings at a time.

\subsection{RF Power}

Illustrate the test setup, clearly indicate measuring distance. Test for various
orientations.

\subsection{RF Range}

Take possibly the best performing device from the previous tests (hopefully the
one with the ducky antenna), and then see how far I can get with the test
modules in various situations. Perhaps test in a corridor with a double
decameter.

\subsection{RF Components}

List recommended components and their BOM price. Compare those with some lower
priced examples (and compare their datasheet specs, like insertion loss). Then
replace those in a design and then run a power test on them, in similar
operating conditions.



