%************************************************
\chapter{Performance tests}\label{ch:performance} % $\mathbb{ZNR}$
%************************************************

After the final prototype has been produced and assembled, we can begin testing
to determine whether or not it meets our requirements. From there, we can decide
which areas to focus on to improve the performance of our device, or if there
are major changes to be made.

This will be followed in \autoref{ch:improvements} by a discussion on what
points could be improved in a future version of this project. 

The two following items are the most important aspects of the device, which we
should test first.

\begin{itemize}
  \item \emph{Power Consumption} : a module must be able to survive on minimal
    power. To determine whether this is possible, we have two concurrent
    approaches : measure the minimum possible current consumption (in a sleep
    state with a wake timer, as well as in deep sleep with a wake interrupt),
    measure the current during a transmission and establish the connection
    between transmission duty cycle and average current consumption. Using data
    gathered from a month-long test using and indoor energy harvesting
    datalogger, we can deduce the maximum duty cycle that can sustain the
    device, which will then allow us to determine which application are suitable
    for this kind of power source.
  \item \emph{RF Power and Range} : Each revision of the PCB will be tested in
    typical working conditions for transmission range and compared with the
    reference designs. We will also take a measurement of the power received at
    a given distance when transmitting a sine wave at various power settings and
    frequencies.
\end{itemize}

\section{Tests}

We will begin by measuring the system's overall power consumption. Since the
basic version 2 of the device contains only a microcontroller, indicator LED and
voltage regulator, we do not expect the results to be very different from the
specifications advertised on the datasheet. These tests will therefore be quite
brief, and serve only to determine whether there are large variations from the
manufacturer's specifications.

\subsection{Current consumption}\label{sub:current-consumption}

Our goal is to determine the average power consumption of the system depending
on the time spent transmitting data, as the radio is the biggest consumer of
power in the device.

The first step therefore is to measure the consumption of the device during
transmission. The rest of the time, the device will either be running in
a lower-power mode with the radio disabled (reading data from sensors for
example) or in sleep mode, waiting to be triggered by an external event.
\autoref{tab:power-radio} shows the consumption measured in three different
modes.

\paragraph{Note} 
all functions described here are part of the ST SimpleMAC library provided
by the manufacturer. The library is proprietary, and the radio is not documented
in the device datasheets. Therefore, the functions provided in this library are
the only way of interacting with the radio portion of the device.

\begin{itemize}
  \item Radio turned off. Using \texttt{ST\_RadioSleep()}

    to turn off the RF part
    of the device.
  \item Radio turned on. The radio element was initialized using the function
    \texttt{ST\_RadioSetPower(3)}, which configures it to operate at its maximum
    standard power setting, \SI{3}{dBm}. The channel was set to 14.
  \item Boost mode. In addition to the standard high power setting, the same
    measurements were made with ``boost mode'' activated. The device datasheet
    claims that this mode can increase receiver sensitivity by up to
    \SI{1}{dBm} and transmit power by up to \SI{0.5}{dBm}.
\end{itemize}

These measurements were done once in idle mode, and once while transmitting
a carrier tone on channel 14 (\SI{2420}{MHz}). This was done by using the
function \texttt{ST\_StartTransmitTone()}.

\begin{table}
  \myfloatalign
  \begin{tabularx}{\textwidth}{X X X}
    \toprule
    \tableheadline{Mode}
    & \tableheadline{Boost on}
    & \tableheadline{Boost off}
    \\ \midrule
    Radio Off           & \SI{10.2}{mA}  & -     \\
    Radio On (standby)  & \SI{29.3}{mA} & \SI{32.5}{mA}   \\
    Transmit tone       & \SI{31.4}{mA} & \SI{35.3}{mA}   \\
    \bottomrule
    \end{tabularx}
    \caption[Radio power consumption]{Radio power consumption}
    \label{tab:power-radio}
\end{table}

These measurements are consistent with the results expected from the
datasheet.\footnote{ More details and consumption for different power modes can
be found in table 150 of the device datasheet, p.215 : \url{
http://www.st.com/internet/com/TECHNICAL_RESOURCES/TECHNICAL_LITERATURE/DATASHEET/CD00248316.pdf
}.} From these results, we can see that it is worth disabling the RF module
of the chip any time it is not required, as it requires twice the amount of
power of the chip core and peripherals. 

\subsubsection{Sleep Mode}\label{sub2:sleep-mode}

Next, we need to know how much power the device consumes in sleep mode, since
this is where the device will be operating over 99\% of the time. 

The SimpleMAC library provided by the manufacturer gives easy access to the
different sleep modes of the STM32W. The following code snippet will turn off
the radio then put the device in \texttt{NOTIMER} sleep mode : this will turn
off all peripherals, including the internal clock, until triggered by an
external interrupt.

\begin{lstlisting}[language=C,caption=Code snippet for deep sleep mode]
  int main(void) {
    while(1) {
      // ... actions here
      ST_RadioSleep();
      halInternalSleep(SLEEPMODE_NOTIMER);

      // ... wait until wake up
    }
  }
\end{lstlisting}

In theory, in this mode, the device should be able to run using less than
\SI{1}{\micro\ampere}. However, in order to achieve this, we must be able to
enter Sleep Modes 1 or 2\footnote{Sleep modes are described in the datasheet,
under section 6.5 \emph{Power Management}, p. 51}.

Unfortunately, in the SWD programming mode does not allow the device to go below
``emulated sleep'', a sleep mode which keeps some parts of the device active to
maintain debug functionalities, such as breakpoints. In our testing, we were not
able to achieve power consumptions below \SI{470}{\micro\ampere}, while the
datasheet suggests a minimal consumption of \SI{800}{nA} in deep sleep with
a wake-up timer active.

After many unsuccessful attemps with various programming tools, we were forced
to give up as time was running out. For the following calculations, we will
assume that it is possible to achieve the manufacturer's advertised consumption,
and round it up to \SI{1}{\micro\ampere}.

\subsubsection{Duty Cycle}\label{sub2:duty-cycle}

Using the consumption data mentioned above, we can determine which proportion of
the time must be spent in sleep mode versus transmission mode to achieve a given
average consumption. To obtain an average current $C$, we would use the
following simplified formula:
\begin{equation*}
  C = (1 - D) S + D T
\end{equation*}

where:

\begin{tabular}[h!]{ll}
$D$   & duty cycle (proportion of time spent transmitting)  \\
$S$   & sleep mode consumption  \\
$T$   & transmission mode consumption \\
\end{tabular}

Which we can rewrite to find the duty cycle:

\begin{equation*}
  D = \frac{C-S}{T-S}
\end{equation*}

As an example, the data gathered during the one-month energy harvesting (see
\autoref{sub2:energy-harvesting}) experiment suggests that we can gather an
average of \SI{10}{\micro\ampere} during one month in December. 

With a sleep mode consumption $S = \SI{1}{\micro{}A}$, transmission current $T = 
\SI{30}{mA}$ and desired consumption $C = \SI{10}{\micro\ampere}$, we need a duty
cycle of 0.03\%, which translates to \SI{18}{ms} of transmission time per
minute.

The amount of data that can be transferred during an interval is not very
straightforward to calculate, as it depends on many parameters and is not
deterministic. The result will depend on the size of each packet, protocol
overhead, but more importantly, the length of the path and number of retries
needed to reach the destination node. However, using some guidelines given by
Digi, who manufacture a range of ZigBee devices\footnote{ \emph{How long does it
take to send data through an 802.15.4 network?} : \url{
http://www.digi.com/support/kbase/kbaseresultdetl?id=3065 }}, we can estimate
that we should be able to send up to two packets of 50 to 100 bytes every
minute. This should be sufficient to send periodic sensor readings or commands
to control remote devices.


\subsubsection{Energy Harvesting}\label{sub2:energy-harvesting}


\begin{table}[h]
  \myfloatalign
  \begin{tabularx}{\textwidth}{X X}
    \toprule
    \tableheadline{Location}
    %& \textsc{average power [\si{\micro\watt}]}
    & \tableheadline{Average Power [\si{\micro\watt}]}
    \\ \midrule
    Hangar, right         & 28.09   \\
    Hangar, left          & 50.83   \\
    Jeremy, right wall    & 151.17  \\
    Jeremy, left wall     & 22.60   \\
    Solaiman, right wall  & 7.38    \\
    Solaiman, left wall   & 23.85   \\
    Lab wall (no sun)     & -2.15   \\
    Lab floor             & 54.50   \\
    \bottomrule
    \end{tabularx}
    \caption[Energy harvesting results]{Energy harvesting results}
    \label{tab:harvest-results}
\end{table}

One of the goals of this project was to determine whether it was possible for
a small sensor device to survive on indoor solar power while providing mesh
networking capabilities with a low data rate. 

Very shortly after the beginning of the project, Texas Instruments released
the BQ25504\footnote{ \url{ http://www.ti.com/product/bq25504 }}, a new chip
dedicated to harvesting energy, mainly from solar panels. Mathieu set out to
design a PCB to test this chip using cheap easy to obtain solar panels.

A summary of the results obtained from a month of energy harvesting can be found
in \autoref{tab:harvest-results}\footnote{ More data, as well as some discussion
of the results, can be found in this blog post : \url{
http://www.limpkin.fr/index.php?post/2012/01/17/Indoor-solar-energy-harvesting\%3A-the-december-numbers
}}

These tests were done using solar panels attached to walls in different parts of
the lab. As we can see, not all locations allow sufficient light to be
collected (one device was placed inside a room illuminated with only artificial
lighting, and no windows. In this case, the energy collected was lower than the
energy lost through leakage.)

These results were collecting during December, which is possibly one of the
worst times for solar energy harvesting. The sun rises late and sets early,
therefore illuminating the panels for a shorter time than the rest of the year.
Use of artificial lighting is also somewhat reduced, as there is less activity
during the winter holidays.

From these results, we can assume that it is possible, given the right
conditions, to collect on average over \SI{20}{\micro\watt} of power. Since the
chip chosen for our application can operate on supply voltages down to
\SI{2.1}{V}, this would translate to an average of \SI{10}{\micro\ampere}
available for our application.

Referring to the calculations in \autoref{sub2:duty-cycle}, this should allow us
to run simple operations, such as taking periodic readings of a temperature or
humidity sensor, sending commands to dim lights depending on ambient light, etc. 

It is worth noting that the chip must not necessarily be fully powered on when
it is not asleep. It may be preferable to wake the chip in a lower-powered mode,
with a slower clock, to read sensors and perform calculations, and only wake the
radio device when data has to be transmitted. We can also aggregate data from
several readings, by recording them to an EEPROM for example, and send several
readings at once, to save power.

\subsection{RF Power}

In the following series of experiments, we will compare the emission power of
our boards with those of the manufacturer's reference kit\footnote{The devices
used are part of the STM32W-RFCKIT
: \url{ http://www.st.com/internet/evalboard/product/251361.jsp }.}

\subsubsection{Technical Details}

\begin{figure}[htb]
  \begin{center}
    \includegraphics[width=1.0\textwidth]{gfx/snippets/spectrum}
  \end{center}
  \caption{802.15.4 Channel Allocation\citep{hunn2010}}
  \label{fig:channel-allocation}
\end{figure}

At \SI{2.4}{GHz}, the spectrum used by the IEEE~802.15.4 standard is divided
into 16 equal channels, each \SI{2}{MHz} wide, spaced \SI{5}{MHz} apart. The
spectrum is centered around \SI{2.45}{GHz}.\citep[pg. 29]{ieee802154}

Channel frequency is defined as $Fc = 2405 + 5 (k-11)$ in Megaherz, for $k
= 11..26$. We will be testing all devices available along the whole spectrum,
testing each channel individually.

\subsubsection{Test setup}

\begin{figure}[htb]
  \begin{center}
    \makebox[\textwidth][c]{
    \includegraphics[width=1\textwidth]{gfx/photos/test-setup}
    }
  \end{center}
  \caption{RF Power Test Setup}
  \label{fig:power-test-setup}
\end{figure}

In order to compare the performance of various antenna choices, we used a simple
test that could be run on all STM32W devices in our posession. Using the
following code, the device under test was made to emit a continuous tone at
a given frequency for a duration of 10 seconds. The average power was measured
using a simple handheld spectrum analyzer\footnote{The device used is the
\emph{RF Explorer}, a cheap piece of equipment which cannot be trusted to give
very accurate results over its whole frequency range. This was our best choice,
since unfortunately our access to professional equipment was limited. However,
we have found that the results were sufficiently repeatable to enable us to
compare different devices \emph{at the same frequency}. More information about
the device can be found here : \url{http://micro.arocholl.com/}.}

\begin{lstlisting}[language=C,caption=Tone generation code]
  while(1) {
    for(i=11; i<27; 1++) {
      blink_green_led();  // signal that we're changing channels

      ST_RadioSetChannel(i);
      ST_RadioStartTransmitTone();

      halCommonDelayMilliseconds(10000); // wait 10 seconds

      ST_RadioStopTransmitTone();
    }
  }
\end{lstlisting}

The results of the first series of tests can be found in
\autoref{tab:power-30cm}. This table lists the power measured on each channel
for various versions of the transmitter. To clarify the column headers, here is
a brief overview of the devices that were analysed:

\begin{itemize}
  \item \emph{v1} Version 1.0 of the PCB, as described in \autoref{sec:v1}.
  \item \emph{v2} Version 2.0 of the PCB, as described in \autoref{sec:v2}.
  \item \emph{usbstick} The USB stick contained in the ST evaluation kit. This
    module uses a chip antenna. See \autoref{fig:eval-kit}.
  \item \emph{rf-mote} This device uses the same reference design as this
    project. It is a battery-operated portable device, and is also part of the
    STM32W-RFCKIT evaluation set.
\end{itemize}

As explained earlier, these measurements were not made with proper test
equipment, nore were they made in a controlled environment. To make relevant
measurements, we would need access to expensive test equiment in an RF anechoic
chamber\footnote{ See \url{ http://en.wikipedia.org/wiki/Anechoic_chamber }.}.

\begin{figure}[bth]
  \makebox[\textwidth][c]{
  \myfloatalign
  \subfloat[USB Stick]
  {\includegraphics[height=.30\linewidth]{gfx/photos/usb-stick}}
  \subfloat[RF-Mote]
  {\includegraphics[height=.30\linewidth]{gfx/photos/rfcmote}}
  }
  \caption{Evaluation kit components}
  \label{fig:eval-kit}
\end{figure}

Unfortunately, the labs who posess the equipment needed were not able to put it
at our disposal. Therefore, we relied on less accurate methods. As can be seen
in the results in \autoref{tab:power-30cm}, the results vary greatly from one
channel to another. This is most likely due to the calibration of the test
device, as it shows across all devices, although some use different antenna
designs. We did note however by performing multiple tests with the same device,
on the same channel, that the repeatability was lower than \SI{1}{dBm}. We can
therefore use these results to compare the performance of different devices on
the same channel, but not one device across several channels.

\subsubsection{Results}\label{power-results}

\begin{table}
  \myfloatalign
  \begin{tabularx}{\textwidth}{c c X X X X}
    \toprule
    \tableheadline{Channel}
        & \tableheadline{Freq [MHz]}
        & \tableheadline{v1}
        & \tableheadline{v2}
        & \tableheadline{usbstick}
        & \tableheadline{rf-mote}
        \\ \midrule
        11       & 2405       & -80   & -83.5 & -85      & -81.5   \\ 
        12       & 2410       & -55.5 & -62.5 & -63.5    & -60     \\ 
        13       & 2415       & -55.5 & -63.5 & -67      & -65.5   \\ 
        14       & 2420       & -73.5 & -79   & -68.5    & -64.5   \\ 
        15       & 2425       & -54.5 & -60   & -60.5    & -56     \\ 
        16       & 2430       & -81   & -87   & -84.5    & -85.5   \\ 
        17       & 2435       & -54   & -62   & -56.5    & -58     \\ 
        18       & 2440       & -60.5 & -70   & -74.5    & -75.5   \\ 
        19       & 2445       & -64.5 & -69   & -56.5    & -58     \\ 
        20       & 2450       & -56.5 & -62   & -55      & -55.5   \\ 
        21       & 2455       & -83.5 & -86   & -79      & -75.5   \\ 
        22       & 2460       & -59   & -61   & -56      & -49     \\ 
        23       & 2465       & -79.5 & -80.5 & -82.5    & -72.5   \\ 
        24       & 2470       & -62   & -62.5 & -62.5    & -49     \\ 
        25       & 2475       & -60   & -60   & -65.5    & -53     \\ 
        26       & 2480       & -82   & -84.5 & -80.5    & -70     \\
        \midrule 
        Average  & -          & -66.3 & -70.8 & -70.9    & -64.3  \\
        \bottomrule
    \end{tabularx}
    \caption[Power Measurements at \SI{30}{cm}]{Power Measurements, in dBm,
    \SI{30}{cm} from emitter}
    \label{tab:power-30cm}
\end{table}


\bigskip
As we can see, version 2.0 of the design performs less well than the first
version, while version 1.0 performs poorer than the rf-mote.

It should be noted that this testing is not very extensive, and will not allow
this device to be compared with standard specifications (range or transmission
power usually given in device datasheets) as the measurements were not made in
a controlled environment according to established procedures. An effort was made
to perform the measurements using the same orientation for the PCB antennas
across devices, but the reality of wireless communication is quite complex, and
results can vary tremendously depending on the orientation of the emitter and
receiver, placement of the devices, as well as the environment. Concrete walls
can reflect signals, and objects in the path may interfere with the reception.

In addition to this, we tested version 2.0 with an external ``ducky antenna''
(\autoref{fig:v2-ducky}). The setup was the same as for the previous devices.
The emitter was placed \SI{30}{cm} away from the portable measuring device, with
the antenna upright. As we can see from the results in
\autoref{tab:power-ducky}, the difference is very significant. Adding the
antenna improves the emission power by \SI{24.7}{dBm} on average compared to the
same device with only the PCB antenna.
For reference, we repeated the measurements at a distance of \SI{2}{m} from the
receiver. These results can be found in the same table. We can notice that the
external antenna has similar results at \SI{2}{m} as the standard device at
\SI{30}{cm}.

\begin{figure}[bth]
  \begin{center}
    \includegraphics[width=.65\linewidth]{gfx/photos/v2-ducky}
  \end{center}
  \caption{v2 device with ``rubber ducky'' antenna}
  \label{fig:v2-ducky}
\end{figure}


\begin{table}
  \myfloatalign
  \begin{tabularx}{0.7\textwidth}{r c X X}
    \toprule
    \tableheadline{Channel}
    & \tableheadline{Freq [MHz]}
    & \tableheadline{\SI{30}{cm}}
    & \tableheadline{\SI{2}{m}}
        \\ \midrule
        11       & 2405       & -60.5      & -86      \\ 
        12       & 2410       & -35.5      & -66      \\ 
        13       & 2415       & -36        & -65.5    \\ 
        14       & 2420       & -58.5      & -89      \\ 
        15       & 2425       & -34        & -65.5    \\ 
        16       & 2430       & -64.5      & -86      \\ 
        17       & 2435       & -35        & -62      \\ 
        18       & 2440       & -41        & -66.5    \\ 
        19       & 2445       & -44        & -67      \\ 
        20       & 2450       & -35.5      & -58      \\ 
        21       & 2455       & -64        & -80.5    \\ 
        22       & 2460       & -34.5      & -55.5    \\ 
        23       & 2465       & -59        & -76.5    \\ 
        24       & 2470       & -38        & -60      \\ 
        25       & 2475       & -36.5      & -58      \\ 
        26       & 2480       & -61.5      & -78      \\
        \midrule
        Average  & -          & -46.1      & -70.0    \\
        \bottomrule
    \end{tabularx}
    \caption[Power Measurements of external antenna]{Power Measurements, in dBm,
    ``ducky'' antenna}
    \label{tab:power-ducky}
\end{table}

\subsection{RF Range}

To add to the comparison of emission power, two modules were taken from the
previous test to evaluate transmission range outdoors. We chose to compare the
best performing device using the reference design (the RFC-Mote) against version
2 of our design, using the external antenna.

\begin{table}[h]
  \myfloatalign
  \begin{tabularx}{0.6\textwidth}{X c}
    \toprule
    \tableheadline{Module Tested}
    & \tableheadline{Range}
    \\ \midrule
    v2 with external antenna  & \SI{41}{m}  \\
    rfc-mote                  & \SI{17}{m}  \\
    \bottomrule
    \end{tabularx}
    \caption[Range test results]{Range test results}
    \label{tab:range-results}
\end{table}

The emitter module was placed on an elevated platform (\SI{1}{m} height),
sending numbered packets periodically,  while the receiver (laptop with USB key
from the development kit) was moved away from the emitter. The distance recorded
in \autoref{tab:range-results} is the maximum distance reached while still
receiving 100\% of packets.

The observations made during this simple test as well as the output power
comparison suggest that it is worth adding an external antenna where possible,
as it has a dramatic effect on performance. Constantly powered router nodes,
which would be connected to the mains power, would be a good candidate for an
external antenna, if space permits.

\section{Comparison Tests}

In parallel to this project, it was discovered that a similar student project
was taking place in a different lab. The eLab\footnote{ \url{
http://elab.epfl.ch/ }} had an ongoing project to develop ANT modules for a home
automation network.\footnote{Student project by Jean-Baptiste Haury, under the
supervision of Fabrizio Lo Conte and Laurent Fabre.}
Since both projects were meant to operate at the same frequencies but use
different antenna designs, we decided to set up a simple experiment to compare,
somewhat informally, the performance of both our designs. 


In this test, illustrated in \autoref{fig:test-setup}, we measure the
transmission of the antenna, using an SMA connector soldered through the board,
as described in \citep{DropoutGuide}.


\begin{figure}[htb]
  \begin{center}
    \makebox[\textwidth][c]{
    \includegraphics[width=1.5\textwidth]{gfx/tests/test02-setup-cropped}
    }
  \end{center}
  \caption{Comparison Test Setup}
  \label{fig:test-setup}
\end{figure}

The network analyzer sweeps a range of frequencies, spanning \SI{0.5}{GHz}
around \SI{2.45}{GHz} and measures the energy received through a ``rubber
ducky'' monopole antenna. Each antenna was tested in two different orientations.
First, the transceiver is placed approximately \SI{50}{cm} away from the
antenna. The antenna is vertical, as in \autoref{fig:test-setup}, secured to
a non-metallic support. The device under test is then placed at the other end of
the setup, its feedline against the support. In this orientation, the plane
containing the PCB is normal to the monopole antenna.

In the second test, the device is placed at the same distance, turned
ninety degrees so
that the antenna is now ``above'' the device. Each test was done in the same way
for both devices under test.

The device developed in this project will be referred to as \textbf{``device A''}.
Jean-Baptiste's device will be referred to as \textbf{``device B''}.


The following screenshots show the results of our four tests.

\begin{figure}[h!]
  \begin{center}
    \includegraphics[width=0.9\textwidth]{gfx/tests/test01-result}
  \end{center}
  \caption{Test 1 : device A, normal}
  \label{fig:test01-result}
\end{figure}

\begin{figure}[h!]
  \begin{center}
    \includegraphics[width=0.9\textwidth]{gfx/tests/test02-result}
  \end{center}
  \caption{Test 2 : device A, parallel}
  \label{fig:test02-result}
\end{figure}

\begin{figure}[h!]
  \begin{center}
    \includegraphics[width=0.9\textwidth]{gfx/tests/test03-result}
  \end{center}
  \caption{Test 3 : device B, normal}
  \label{fig:test03-result}
\end{figure}

\begin{figure}[h!]
  \begin{center}
    \includegraphics[width=0.9\textwidth]{gfx/tests/test04-result}
  \end{center}
  \caption{Test 4 : device B, parallel}
  \label{fig:test04-result}
\end{figure}

\pagebreak

The results are summarized in \autoref{tab:comparison-results}. This table shows
the attenuation measured in each case, in the center of the operating bandwidth
(\ie~\SI{2.45}{GHz})

It is worth noting at this point that these results are not to be taken for
their quantitative values. Such testing requires a strict methodology in
a controlled environment, which was not available in the context of this work.
Instead, we present these results simply as a matter of comparison between two
similar projects, using two different choices of antenna designs.

\begin{table}
    \myfloatalign
  \begin{tabularx}{0.6\textwidth}{Xcc} \toprule
    \tableheadline{} & \tableheadline{normal}
    & \tableheadline{parallel} \\ \midrule
    device A  & \SI{-63.3}{dB}  & \SI{-53.0}{dB} \\
    device B  & \SI{-51.1}{dB}  & \SI{-43.1}{dB} \\
    \bottomrule
  \end{tabularx}
  \caption[Comparison results]{Comparison results}
  \label{tab:comparison-results}
\end{table}

\subsection{Observations}

From this test, we can see that even though this project has achieved
performance similar to the reference design, it is possible to do better, given
the same requirements. It is quite clear from the approximately \SI{10}{dB}
difference between both designs, that we can expect better results from using
a different antenna in future versions.

It remains to be seen however if it is possible to improve the design using only
a PCB trace antenna, or if we must balance this potential performance
improvement with the cost of a chip antenna.

Although in theory these measurements were taken from the same point in the
signal path, our device uses a band-pass filter in order to attenuate noise
outside of the transmission band. This is often necessary to pass certification
tests, as there is a limit to the amounts of radiation that can be produced
outside of the legal frequency range. The band-pass filter has an insertion loss
of up to \SI{3}{dB}, which must be taken into account. Even so, device B still
outperforms our own.

Unfortunately, we do not know for a fact whether such a protection is necessary
or not, as it would require equipment and qualifications beyond our reach. 

\begin{figure}[htb]
  \begin{center}
    \includegraphics[width=0.9\textwidth]{gfx/radiation-pattern}
  \end{center}
  \caption{PCB Antenna Radiation Pattern\citep{AN3359}}
  \label{fig:radiation-pattern}
\end{figure}

Our final observation is that both devices are very sensitive to changes in
orientation. Both antennas are meant to be omni-directional, however, some
directions are clearly favoured over others. If we refer to page 12 of the
antenna design guide\citep{AN3359}, we can see that the first orientation tested
places the receiving antenna in the worst possible location in the radiation
pattern. For convenience, this pattern is reproduced in
\autoref{fig:radiation-pattern}

