%************************************************
\chapter{Future improvements}\label{ch:improvements}
%************************************************

Here we'll discuss some compromises that were made. Notably:

\begin{itemize}
  \item Lack of knowledge in RF design, so we went with the reference design for
    the chip.
  \item Time constraints, needed to push through with the design, otherwise it
    would have been better to test various different antenna designs. 
  \item Choosing a different chip : if the requirements had been a little
    different, we could have chosen a different solution, which would have been
    cheaper, more powerful and/or more energy efficient.
\end{itemize}


TODO : integrate this info into this chapter

What we've deduced from the project : was it the right choice? What could we
have done better?

\begin{itemize}
  \item \emph{Power consumption} : original specifications called for an audio
    amplification stage, which would have used tens of micro amperes. As
    a reference, the MAX4468 micro power amplifier which was considered for this
    application and used in the first version of the PCB requires between
    \SIrange{24}{50}{\micro\ampere}. Compared to this, a difference between
    400nA and 20nA was not significant. However, if the system is designed to be
    completely turned off most of the time, then a combination of an ultra low
    power microcontroller with a CC2420 or atmel chip would be more efficient.
    However, even the most power-efficient devices on the market currently offer
    % TODO : actually 700 to 1200 nA in RTC sleep
    performance similar to the STM32W. Texas Instruments' MSP430F2001 for
    example requires \SI{500}{nA} in idle mode\footnote{ \url{
    http://www.ti.com/product/msp430f2001 }}, which is in the same order of
    magnitude as the \SI{800}{nA} quoted for the STM32W108. In this case, the
    duty cycle between idle and transmitting is more important than the power
    savings from one chip to another.
  \item \emph{Antenna diversity} : performance in the \SI{2.4}{GHz} band is pretty
    random (show some examples!). Very sensitive to positioning and orientation.
    Some chips now offer antenna diversity (link to very recent Atmel chip),
    which could be a very interesting option if space permits. Especially if
    using PCB antenna, would not increase costs too much.
\end{itemize}

\subsection{RF Components}

TODO : explain how we should do this, with proper test equipment
List recommended components and their BOM price. Compare those with some lower
priced examples (and compare their datasheet specs, like insertion loss). Then
replace those in a design and then run a power test on them, in similar
operating conditions.

% TODO : find a place to talk about this.
RF Components : Some components have a very high influence on the
    total cost of the device. We shall test alternatives to the recommended
    components to determine if we can lower the cost without compromising
    performance.




\section{Design Issues}

Talk about the design itself. Problem with this project is that it's only
produced a proof of concept device. It would be more interesting to have a whole
family of devices that can perform various functions, gather sensor data etc.

If this project were to be developed further, the first step would be to create
some add-on modules to be connected to the expansion port. This however is quite
wasteful in space and monies (connectors are expensive, and the current choice
is quite large). Ideally, once a device has been developed and tested
satisfactorily, the add-on module should be incorporated onto one PCB with the
transmitter and become a standalone module with one specific application.

Also, note that this way of doing things (since we're using a SoC) is a little
bit backwards from the way things tend to be done : the ``base unit'' is just
a chip with an antenna, and we plug modules into it to add functionality.
Usually, you would build a platform for one or more applications, and then strap
the RF device to it to add RF functionality, not make it the core of the
application.

Of course, there's nothing to prevent us from doing that : i.e. developing
a 6LoWPAN XBee clone (personal project for my future!) but it's probably going
to be more expensive than necessary!

\bigskip
 
\noindent\textit{\myLocation, \myTime}

\smallskip

\begin{flushright}
    \begin{tabular}{m{5cm}}
        \\ \hline
        \centering\myName \\
    \end{tabular}
\end{flushright}
