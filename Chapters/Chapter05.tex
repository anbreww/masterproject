%************************************************
\chapter{Future improvements}\label{ch:improvements}
%************************************************

During this project, many compromises were made. In this chapter, we will
discuss some of the problems we encoutered and choices that were made. We will
conclude this report by suggesting possible ways to improve the current design,
or what to change if the design is to be re-done from scratch.

\section{Antenna}

The first item to note is that due to our lack of knowledge in the design of RF
systems, we decided to use the reference design provided by the manufacturer. An
effort was made to replicate this design exactly, down to the placement of
passive elements and vias, as it had been suggested that this could impact the
performance significantly. As the first version showed, we obtained very similar
results to the reference.

In the second version, we introduced some slight modifications to the layout of
the RF components, and saw the performance decrease. Whether this is directly
linked to the placement of the components, we do not know for sure.

Knowing this, we believe it would be smarter to produce several versions of each
PCB, every time with slight variations in the design of the antenna, and compare
each version, refining the design by successive iterations. At the beginning,
a lot of time was intentionally spent examining various platform choices in
order to make an educated decision. As a result, the actual design was slightly
rushed, and not enough time was spent on testing various antenna designs.
Instead, we opted to produce one prototype rapidly, sticking to the reference as
closely as possible, to ensure that everything functioned as expected.

In hindsight, knowing now that there were no issues with the design, it may not
have caused much harm to delay production of the first PCB by one or two weeks,
to allow us to test several antenna shapes at the same time. If the design had
been unsuccessful though, this extra time would most likely have been wasted.


During our experiments, we noticed that performance in the \SI{2.4}{GHz} band is
quite random, and can vary tremendously with only very little movement.
Sometimes as little as a centimeter difference in placement can have
a noticeable impact on a device's reception. Some devices provide a solution to
this, called \emph{antenna diversity} : one device has two or more antennas,
connected to the transceiver via an RF multiplexer. The transceiver switches
between the antennas periodically, and automatically selects the antenna with
the best reception.\footnote{The AT86RF231 is an example of a chip with
a built-in antenna diversity feature : \url{
http://www.atmel.com/dyn/resources/prod_documents/doc8158.pdf }.}

\section{Microcontroller}

Our choice of microcontroller was very much dictated by a small subset of
specifications, most notably the need for certain analog functions and
processing power. In a more general case of a sensor network design, the
requirements may be slightly different, and allow a different choice.

Had the requirements been looser, we may have opted for a different chip, as
some platforms are cheaper than the one chosen for this project.

Our choice to use a System-on-Chip rather than two separate chips to save space
and simplify the board design is not necessarily the most economical, as the
same functionality could have been achieved with a (cheaper) transceiver chip
and separate processing core. For example, a combination of the STM32F100 Cortex
M3 microcontroller with the AT86RF230 mentioned in
\autoref{tab:transceiver-comparison}, would have cost approximately \$1 less
than the STM32W108, so long as the memory was sufficient. Considering the
processing power and memory available on the STM32W108, it provides a very good
value compared to a two-chip solution.

\section{Power consumption}

The specifications of this project called for an audio amplification stage which
would be permanently powered and wake the chip from sleep when the audio signal
went over a certain threshold. This feature, which was integrated to the first
version of the PCB, requires tens of micro amperes. As a reference, the MAX4458
micro power amplifier used in this design requires
\SIrange{24}{50}{\micro\ampere}.

In the tables from \autoref{ch:appendix-comparison}, we will note that some
transceivers only require \SI{20}{nA} when not activated, which is significantly
less than the \SI{400}{nA} that our STM32W requires. However, compared to the
consumption of the audio stage, this difference is negligible. Furthermore, the
use of a separate transceiver would still require a second microcontroller to be
incorporated into the design.

Texas instruments advertise the MSP430F2001 as being the most power-efficient
microcontroller on the market.\footnote{ \url{
http://www.ti.com/product/msp430f2001 } } Nevertheless, this chip requires
\SI{500}{nA} of power in idle mode, which is in the same order of magnitude as
the \SI{800}{nA} quoted for the STM32W108. From the data and calculations put
together in the previous chapter, we have noticed that the duty cycle between
the system's idle time and transmission time has a larger impact on overall
power consumption than the power savings between one chip or the other.

\subsection{RF Components}



TODO : explain how we should do this, with proper test equipment
List recommended components and their BOM price. Compare those with some lower
priced examples (and compare their datasheet specs, like insertion loss). Then
replace those in a design and then run a power test on them, in similar
operating conditions.

% TODO : find a place to talk about this.
RF Components : Some components have a very high influence on the
    total cost of the device. We shall test alternatives to the recommended
    components to determine if we can lower the cost without compromising
    performance.




\section{Design Issues}

Talk about the design itself. Problem with this project is that it's only
produced a proof of concept device. It would be more interesting to have a whole
family of devices that can perform various functions, gather sensor data etc.

If this project were to be developed further, the first step would be to create
some add-on modules to be connected to the expansion port. This however is quite
wasteful in space and monies (connectors are expensive, and the current choice
is quite large). Ideally, once a device has been developed and tested
satisfactorily, the add-on module should be incorporated onto one PCB with the
transmitter and become a standalone module with one specific application.

Also, note that this way of doing things (since we're using a SoC) is a little
bit backwards from the way things tend to be done : the ``base unit'' is just
a chip with an antenna, and we plug modules into it to add functionality.
Usually, you would build a platform for one or more applications, and then strap
the RF device to it to add RF functionality, not make it the core of the
application.

Of course, there's nothing to prevent us from doing that : i.e. developing
a 6LoWPAN XBee clone (personal project for my future!) but it's probably going
to be more expensive than necessary!

\bigskip
 
\noindent\textit{\myLocation, \myTime}

\smallskip

\begin{flushright}
    \begin{tabular}{m{5cm}}
        \\ \hline
        \centering\myName \\
    \end{tabular}
\end{flushright}
