%************************************************
\chapter{Future improvements}\label{ch:improvements}
%************************************************

During this project, many compromises were made. In this chapter, we will
discuss some of the problems we encoutered and choices that were made. We will
conclude this report by suggesting possible ways to improve the current design,
or what to change if the design is to be re-done from scratch.

\section{Antenna}

The first item to note is that due to our lack of knowledge in the design of RF
systems, we decided to use the reference design provided by the manufacturer. An
effort was made to replicate this design exactly, down to the placement of
passive elements and vias, as it had been suggested that this could impact the
performance significantly. As the first version showed, we obtained very similar
results to the reference.

In the second version, we introduced some slight modifications to the layout of
the RF components, and saw the performance decrease. Whether this is directly
linked to the placement of the components, we do not know for sure.

Knowing this, we believe it would be smarter to produce several versions of each
PCB, every time with slight variations in the design of the antenna, and compare
each version, refining the design by successive iterations. At the beginning,
a lot of time was intentionally spent examining various platform choices in
order to make an educated decision. As a result, the actual design was slightly
rushed, and not enough time was spent on testing various antenna designs.
Instead, we opted to produce one prototype rapidly, sticking to the reference as
closely as possible, to ensure that everything functioned as expected.

In hindsight, knowing now that there were no issues with the design, it may not
have caused much harm to delay production of the first PCB by one or two weeks,
to allow us to test several antenna shapes at the same time. If the design had
been unsuccessful though, this extra time would most likely have been wasted.


During our experiments, we noticed that performance in the \SI{2.4}{GHz} band is
quite random, and can vary tremendously with only very little movement.
Sometimes as little as a centimeter difference in placement can have
a noticeable impact on a device's reception. Some devices provide a solution to
this, called \emph{antenna diversity} : one device has two or more antennas,
connected to the transceiver via an RF multiplexer. The transceiver switches
between the antennas periodically, and automatically selects the antenna with
the best reception.\footnote{The AT86RF231 is an example of a chip with
a built-in antenna diversity feature : \url{
http://www.atmel.com/dyn/resources/prod_documents/doc8158.pdf }.}

\section{Microcontroller}

Our choice of microcontroller was very much dictated by a small subset of
specifications, most notably the need for certain analog functions and
processing power. In a more general case of a sensor network design, the
requirements may be slightly different, and allow a different choice.

Had the requirements been looser, we may have opted for a different chip, as
some platforms are cheaper than the one chosen for this project.

Our choice to use a System-on-Chip rather than two separate chips to save space
and simplify the board design is not necessarily the most economical, as the
same functionality could have been achieved with a (cheaper) transceiver chip
and separate processing core. For example, a combination of the STM32F100 Cortex
M3 microcontroller with the AT86RF230 mentioned in
\autoref{tab:transceiver-comparison}, would have cost approximately \$1 less
than the STM32W108, so long as the memory was sufficient. Considering the
processing power and memory available on the STM32W108, it provides a very good
value compared to a two-chip solution.

\section{Power consumption}

The specifications of this project called for an audio amplification stage which
would be permanently powered and wake the chip from sleep when the audio signal
went over a certain threshold. This feature, which was integrated to the first
version of the PCB, requires tens of micro amperes. As a reference, the MAX4458
micro power amplifier used in this design requires
\SIrange{24}{50}{\micro\ampere}.

In the tables from \autoref{ch:appendix-comparison}, we will note that some
transceivers only require \SI{20}{nA} when not activated, which is significantly
less than the \SI{400}{nA} that our STM32W requires. However, compared to the
consumption of the audio stage, this difference is negligible. Furthermore, the
use of a separate transceiver would still require a second microcontroller to be
incorporated into the design.

Texas instruments advertise the MSP430F2001 as being the most power-efficient
microcontroller on the market.\footnote{ \url{
http://www.ti.com/product/msp430f2001 } } Nevertheless, this chip requires
\SI{500}{nA} of power in idle mode, which is in the same order of magnitude as
the \SI{800}{nA} quoted for the STM32W108. From the data and calculations put
together in the previous chapter, we have noticed that the duty cycle between
the system's idle time and transmission time has a larger impact on overall
power consumption than the power savings between one chip or the other.

\section{RF Components}

The reference design lists recommended components for the RF path which are
expensive, given the cost of the overall system. We had originally planned to
source several alternative and less expensive components, and measure the
frequency response of the system with different combinations of components. 

The components considered are the balun and RF filter. \autoref{tab:baluns}
shows a comparison of baluns, while \autoref{tab:filters} compares three
possible choices for RF filters.

\begin{table}[h]
  \myfloatalign
  \begin{tabularx}{\textwidth}{X X c}
    \toprule
    \tableheadline{Digikey part \#}
    & \tableheadline{Manufacturer}
    & \tableheadline{Price}
    \\ \midrule
    732-2230-1-ND & Würth Elektronik &  \$ 1.83 \\
    445-3985-1-ND & TDK  & \$ 1.03  \\
    712-1045-1-ND & Johanson & \$ 0.54 \\
    712-1041-1-ND & Johanson & \$ 0.33 \\
    \bottomrule
    \end{tabularx}
    \caption[RF Baluns]{RF Baluns}
    \label{tab:baluns}
\end{table}

The first component in \autoref{tab:filters} is the recommended component. It
was not available from the usual online retailers (Digi-key, Farnell, Mouser,
\ldots) and had to be sampled from the manufacturer. The price is unknown as it
is not listed on the manufacturer's website.

\begin{table}[h]
  \myfloatalign
  \begin{tabularx}{\textwidth}{p{2.6cm} p{2.6cm} X X}
    \toprule
    \tableheadline{Digikey part \#}
    & \tableheadline{Manufacturer}
    & \tableheadline{Insertion loss}
    & \tableheadline{Price}
    \\ \midrule
    748 421 245* & Würth Elek. & 2.5dB  & \$ - \\
    712-1089-1-ND & Johanson & 1.5dB & \$ 0.48 \\
    712-1092-1-ND & Johanson & 1.2dB & \$ 0.59 \\
    \bottomrule
    \end{tabularx}
    \caption[RF Filters]{RF Filters}
    \label{tab:filters}
\end{table}

The above tables indicate that it is possible to save a significant amount of
money by using alternative components, as well as improve the performance : some
alternative RF filters offer a lower insertion loss.

However, since the expected variations are quite low (around
\SIrange{1}{2}{dB}), a comparison like this would necessitate better test
equipment than we had at our disposal, as well as a controlled test environment
to ensure the repeatability of our results. We believe the results of this
comparison would be interesting, if only to show that the difference in
performance may not warrant such a large price difference.


\section{Conclusion}

Overall, the design developed throughout this project has met the requirements
that were given in the beginning, and has resulted in a functional prototype
which could be used in a wide variety of projects. As a result of the way this
project was structured, the end result is mostly a proof of concept, which is
not necessarily very interesting on a functional level.

If this project were to be developed further, and to make it more appealing, the
first step would be to create some add-on modules to plug in to the expansion
port. This was not part of the plan for the project, but it would be interesting
to develop a small family of devices that can communicate together and perform
various functions, react to user input, gather sensor data, control devices
remotely, etc.

Considering the size of the device and the connector, this approach can be quite
wasteful both in size and financially. Once a module has been developed that
fulfills a given function, both the functionality of the module and the RF
section could be integrated into the same PCB design, resulting in one compact
application-specific device.

Another particularity to be noted about this device is that it is somewhat
backwards from the ``usual'' way of doing things. In most cases, one would
develop a solution to perform a given function, then add a module that provides
wireless connectivity. In this case, we are doing the opposite : the main
processor is part of the RF module, and additional functionality is to be
provided as a module. 

Naturally, there is nothing to prevent this module from being reworked into
a device that would offer wireless connectivity as an add-on, much like XBee
devices from Digi. Given the impending popularisation of IPv6\footnote{Many
large internet companies will be turning on IPv6 support for good on June 6\th
2012 : \url{ http://www.worldipv6launch.org/}.}, an XBee-like device providing
     6LoWPAN connectivity is likely to be quite successful in the near future.
\\[1.5cm]
 
\noindent\textit{\myLocation, \myTime}

\smallskip

\begin{flushright}
    \begin{tabular}{m{5cm}}
        \\ \hline
        \centering\myName \\
    \end{tabular}
\end{flushright}
