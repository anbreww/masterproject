%************************************************
\chapter{Future improvements}\label{ch:improvements}
%************************************************

Here we'll discuss some compromises that were made. Notably:

\begin{itemize}
  \item Lack of knowledge in RF design, so we went with the reference design for
    the chip.
  \item Time constraints, needed to push through with the design, otherwise it
    would have been better to test various different antenna designs. 
  \item Choosing a different chip : if the requirements had been a little
    different, we could have chosen a different solution, which would have been
    cheaper, more powerful and/or more energy efficient.
\end{itemize}

\section{Design Issues}

Talk about the design itself. Problem with this project is that it's only
produced a proof of concept device. It would be more interesting to have a whole
family of devices that can perform various functions, gather sensor data etc.

If this project were to be developed further, the first step would be to create
some add-on modules to be connected to the expansion port. This however is quite
wasteful in space and monies (connectors are expensive, and the current choice
is quite large). Ideally, once a device has been developed and tested
satisfactorily, the add-on module should be incorporated onto one PCB with the
transmitter and become a standalone module with one specific application.

Also, note that this way of doing things (since we're using a SoC) is a little
bit backwards from the way things tend to be done : the ``base unit'' is just
a chip with an antenna, and we plug modules into it to add functionality.
Usually, you would build a platform for one or more applications, and then strap
the RF device to it to add RF functionality, not make it the core of the
application.

Of course, there's nothing to prevent us from doing that : i.e. developing
a 6LoWPAN XBee clone (personal project for my future!) but it's probably going
to be more expensive than necessary!
